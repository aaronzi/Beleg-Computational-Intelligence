\section{Theoretische Grundlagen} \label{sec:theoretische_grundlagen}
Im ersten Schritt werden die einzelnen Layer eines CNNs mittels Beispielen erläutert. Anschließend erfolgt die Aufbereitung des AlexNets, welches diese Schichten anwendet und als Basis für die Bildverarbeitung zur Kollisionsvermeidung dienen wird. Im letzten Schritt wird auf das Transfer-Learning, sowie dessen Vor- und Nachteile, aber auch dessen Nutzen für die Arbeit eingegangen.

\subsection{Convolutional Neural Networks (CNN)} \label{sec: cnn}
Ein \textbf{CNN} ist ein \textbf{künstliches neuronales Netz}, welches aus mehreren Schichten (Layern) besteht und Faltungseigenschaften anwendet \cite{Wuertz2019}. Die verschiedenen Layer werden in vier Kategorien eingeteilt, welche in den nachfolgenden Kapiteln erläutert werden.
\begin{itemize}
    \item Convolution (Faltung)
    \item Pooling (Zusammenlegung)
    \item Rectification (Gleichrichtung).
    \item Fully-Connected (Vollständig verbunden)
\end{itemize}

Zur schematischen Übersicht des Aufbaus dient \autoref{fig:Bild2.1}. Da der Recitifcation-Layer nicht immer Anwendung findet, ist dieser nicht in der Abbildung enthalten.\\
Das CNN wird durch große Datenmengen, deren Eigenschaften und Klassen bekannt sind, vortrainiert. Das Eingangsmedium wird durch unterschiedliche \textbf{Filtertypen} auf bestimmte Merkmale untersucht. Die Ergebnisse werden durch \textbf{mathematische Operationen} gewichtet und in s.g. \textbf{Feature-Maps} gespeichert \cite{Wuertz2019}. Anschließend erfolgt in der \textbf{Klassifizierung}, anhand der Gewichtung, die Zuordnung zu einer vorgegebenen Klasse. Die Zuordnung wird anschließend auf Richtigkeit geprüft, um Korrekturen vorzunehmen. Die Eingangsbilder werden mit kleinen Änderungen (\autoref{fig:Bild2.2}) eingegeben, um weitere Feauture-Maps zu erzeugen. Dies erfolgt in Form von Feedbackschleifen (Epochen) \cite{Wuertz2019}. Dieser Prozess wird als \glqq\textbf{Trainieren des neuronalen Netzes}\grqq\:bezeichnet. Wird das trainierte Netz auf unbekannte, jedoch ähnliche Eingangsmedien angewendet, werden die Filter der Merkmale nicht weiter angepasst. Dieser Vorgang heißt \glqq\textbf{Anwendung des neuronalen Netzes} \grqq\:\cite{Wuertz2019}.

\begin{figure}[H]
    \centering
    \scalebox{0.8}
    {
    \begin{tikzpicture}[framed][domain=0:0]
    % Größe der Bildumgebung
        \draw[black] (0, 0) rectangle (13.75, 6.25);
    
    % Eingangsbild
        \draw[black] (0.5, 3) rectangle (2, 4.5);
        
    % 1. Layer
        \draw[black] (5, 2.25) rectangle (6.5, 3.75);
        \draw[black] (5, 2.5) -- ++(-0.25, 0) -- ++(0, 1.5) -- ++(1.5, 0) -- ++(0, -0.25);
        \draw[black] (4.75, 2.75) -- ++(-0.25, 0) -- ++(0, 1.5) -- ++(1.5, 0) -- ++(0, -0.25);
        \draw[black] (4.5, 3) -- ++(-0.25, 0) -- ++(0, 1.5) -- ++(1.5, 0) -- ++(0, -0.25);
        \draw[black] (4.25, 3.25) -- ++(-0.25, 0) -- ++(0, 1.5) -- ++(1.5, 0) -- ++(0, -0.25);
        \draw[black] (4, 3.50) -- ++(-0.25, 0) -- ++(0, 1.5) -- ++(1.5, 0) -- ++(0, -0.25);
        \draw[black] (3.75, 3.75) -- ++(-0.25, 0) -- ++(0, 1.5) -- ++(1.5, 0) -- ++(0, -0.25);
        \draw[black] (3.5, 4) -- ++(-0.25, 0) -- ++(0, 1.5) -- ++(1.5, 0) -- ++(0, -0.25);
    
    % 2. Layer
        \draw[black] (8.5, 2.5) rectangle (9.5, 3.5);
        \draw[black] (8.5, 2.9) -- ++(-0.4, 0) -- ++(0, 1) -- ++(1, 0) -- ++(0, -0.4);
        \draw[black] (8.1, 3.3) -- ++(-0.4, 0) -- ++(0, 1) -- ++(1, 0) -- ++(0, -0.4);
        \draw[black] (7.7, 3.7) -- ++(-0.4, 0) -- ++(0, 1) -- ++(1, 0) -- ++(0, -0.4);
        \draw[black] (7.3, 4.1) -- ++(-0.4, 0) -- ++(0, 1) -- ++(1, 0) -- ++(0, -0.4);
    
    % 3. Layer
        \draw[black] (11, 5) circle (0.15cm);
        \draw[black] (11, 4.25) circle (0.15cm);
        \draw[black] (11, 3.5) circle (0.15cm);
        \draw[black] (11, 2.75) circle (0.15cm);
        \draw[black] (11, 2) circle (0.15cm);
        
    % Ausgabe
        \draw[black] (13, 4.25) circle (0.15cm);
        \draw[black] (13, 3.75) circle (0.15cm);
        \draw[black] (13, 3.25) circle (0.15cm);
    
    % Verbindungen
        \draw[black] (1.5, 4) rectangle (1.75, 4.25);
        \draw[black] (6, 3.25) rectangle (6.25, 3.5);
        \draw[black] (9.2, 3.2) rectangle (9.35, 3.35);
        \draw[black, dashed] (1.75, 4.25) -- (5.75, 3);
        \draw[black, dashed] (1.75, 4) -- (5.75, 3);
        \draw[black, dashed] (6.25, 3.5) -- (9, 3);
        \draw[black, dashed] (6.25, 3.25) -- (9, 3);
        \draw[black, dashed] (7.9, 5.1) -- (10.85, 5);
        \draw[black, dashed] (9.5, 2.5) -- (10.85, 2);
    
    % Fully-Connected Verbindungen
        \draw[black, dashed] (11.15, 5) -- (12.85, 4.25);
        \draw[black, dashed] (11.15, 5) -- (12.85, 3.75);
        \draw[black, dashed] (11.15, 5) -- (12.85, 3.25);
        \draw[black, dashed] (11.15, 4.25) -- (12.85, 4.25);
        \draw[black, dashed] (11.15, 4.25) -- (12.85, 3.75);
        \draw[black, dashed] (11.15, 4.25) -- (12.85, 3.25);
        \draw[black, dashed] (11.15, 3.5) -- (12.85, 4.25);
        \draw[black, dashed] (11.15, 3.5) -- (12.85, 3.75);
        \draw[black, dashed] (11.15, 3.5) -- (12.85, 3.25);
        \draw[black, dashed] (11.15, 2.75) -- (12.85, 4.25);
        \draw[black, dashed] (11.15, 2.75) -- (12.85, 3.75);
        \draw[black, dashed] (11.15, 2.75) -- (12.85, 3.25);
        \draw[black, dashed] (11.15, 2) -- (12.85, 4.25);
        \draw[black, dashed] (11.15, 2) -- (12.85, 3.75);
        \draw[black, dashed] (11.15, 2) -- (12.85, 3.25);
        
    % Bezeichnungen
        \draw[decorate , decoration = {brace, amplitude = 10pt, mirror}] (0.5,1.25) -- ++(1.5, 0) node[black, midway, xshift = 0cm, yshift = -0.6cm] {\footnotesize $Eingangsbild$};
        \draw[decorate , decoration = {brace, amplitude = 10pt, mirror}] (3.5,1.25) -- (9.5, 1.25) node[black, midway, xshift = 0cm, yshift = -0.6cm] {\footnotesize $Analyse\:der\:Merkmale$};
        \draw[decorate , decoration = {brace, amplitude = 10pt, mirror}] (10.75, 1.25) -- (13.25, 1.25) node[black, midway, xshift = 0cm, yshift = -0.6cm] {\footnotesize $Klassifizierung$};
    
    % Bezeichnung der Layer
        \node[black, scale = 0.6, font = \bfseries] at (1.25, 5.75) {Eingang};
        \node[black, scale = 0.6, font = \bfseries] at (4.8625, 5.75) {Convolution};
        \node[black, scale = 0.6, font = \bfseries] at (8.2, 5.75) {Pooling};
        \node[black, scale = 0.6, font = \bfseries] at (11, 5.75) {Fully-Connected};
        \node[black, scale = 0.6, font = \bfseries] at (13, 5.75) {Ausgang};
    \end{tikzpicture}
    }
    \caption[Basisstruktur eines CNN]{Basisstruktur eines CNN \cite{Borchers2022}}
    \label{fig:Bild2.1}
\end{figure}

\begin{figure}[H]
    \centering
    \scalebox{0.9}
    {
    \begin{tikzpicture}[framed][domain=0:0]
    % Größe der Bildumgebung
        \draw[black] (0, 0) rectangle (13, 7);
    
    % Größe des Eingangsbilds 1
        \draw[black] (0.5, 1.75) rectangle (5.0, 6.25);
    
    % Vertikale Linien des Eingangsbilds 1
        \draw[black, thin] (1.0, 1.75) -- ++(0, 4.5);
        \draw[black, thin] (1.5, 1.75) -- ++(0, 4.5);
        \draw[black, thin] (2.0, 1.75) -- ++(0, 4.5);
        \draw[black, thin] (2.5, 1.75) -- ++(0, 4.5);
        \draw[black, thin] (3.0, 1.75) -- ++(0, 4.5);
        \draw[black, thin] (3.5, 1.75) -- ++(0, 4.5);
        \draw[black, thin] (4.0, 1.75) -- ++(0, 4.5);
        \draw[black, thin] (4.5, 1.75) -- ++(0, 4.5);
    
    % Horizontale Linien des Eingangsbilds 1
        \draw[black, thin] (0.5, 2.25) -- ++(4.5, 0);
        \draw[black, thin] (0.5, 2.75) -- ++(4.5, 0);
        \draw[black, thin] (0.5, 3.25) -- ++(4.5, 0);
        \draw[black, thin] (0.5, 3.75) -- ++(4.5, 0);
        \draw[black, thin] (0.5, 4.25) -- ++(4.5, 0);
        \draw[black, thin] (0.5, 4.75) -- ++(4.5, 0);
        \draw[black, thin] (0.5, 5.25) -- ++(4.5, 0);
        \draw[black, thin] (0.5, 5.75) -- ++(4.5, 0);
    
    % Größe des Eingangsbilds 2
        \draw[black] (8, 1.75) rectangle (12.5, 6.25);
    
    % Vertikale Linien des Eingangsbilds 2
        \draw[black, thin] (8.5, 1.75) -- ++(0, 4.5);
        \draw[black, thin] (9, 1.75) -- ++(0, 4.5);
        \draw[black, thin] (9.5, 1.75) -- ++(0, 4.5);
        \draw[black, thin] (10, 1.75) -- ++(0, 4.5);
        \draw[black, thin] (10.5, 1.75) -- ++(0, 4.5);
        \draw[black, thin] (11, 1.75) -- ++(0, 4.5);
        \draw[black, thin] (11.5, 1.75) -- ++(0, 4.5);
        \draw[black, thin] (12, 1.75) -- ++(0, 4.5);
    
    % Horizontale Linien des Eingangsbilds 2
        \draw[black, thin] (8, 2.25) -- ++(4.5, 0);
        \draw[black, thin] (8, 2.75) -- ++(4.5, 0);
        \draw[black, thin] (8, 3.25) -- ++(4.5, 0);
        \draw[black, thin] (8, 3.75) -- ++(4.5, 0);
        \draw[black, thin] (8, 4.25) -- ++(4.5, 0);
        \draw[black, thin] (8, 4.75) -- ++(4.5, 0);
        \draw[black, thin] (8, 5.25) -- ++(4.5, 0);
        \draw[black, thin] (8, 5.75) -- ++(4.5, 0);
    
    % Zahlen positionieren im Eingangsbild 1
        % 1. Spalte
        \node[red, scale = 0.6, font = \bfseries] at (0.75, 2) {-1};
        \node[red, scale = 0.6, font = \bfseries] at (0.75, 2.5) {-1};
        \node[red, scale = 0.6, font = \bfseries] at (0.75, 3) {-1};
        \node[red, scale = 0.6, font = \bfseries] at (0.75, 3.5) {-1};
        \node[red, scale = 0.6, font = \bfseries] at (0.75, 4) {-1};
        \node[red, scale = 0.6, font = \bfseries] at (0.75, 4.5) {-1};
        \node[red, scale = 0.6, font = \bfseries] at (0.75, 5) {-1};
        \node[red, scale = 0.6, font = \bfseries] at (0.75, 5.5) {-1};
        \node[red, scale = 0.6, font = \bfseries] at (0.75, 6) {-1};
        
        % 2. Spalte
        \node[red, scale = 0.6, font = \bfseries] at (1.25, 2) {-1};
        \node[black, scale = 0.6, font = \bfseries] at (1.25, 2.5) {1};
        \node[red, scale = 0.6, font = \bfseries] at (1.25, 3) {-1};
        \node[red, scale = 0.6, font = \bfseries] at (1.25, 3.5) {-1};
        \node[red, scale = 0.6, font = \bfseries] at (1.25, 4) {-1};
        \node[red, scale = 0.6, font = \bfseries] at (1.25, 4.5) {-1};
        \node[red, scale = 0.6, font = \bfseries] at (1.25, 5) {-1};
        \node[black, scale = 0.6, font = \bfseries] at (1.25, 5.5) {1};
        \node[red, scale = 0.6, font = \bfseries] at (1.25, 6) {-1};
        
        % 3. Spalte
        \node[red, scale = 0.6, font = \bfseries] at (1.75, 2) {-1};
        \node[red, scale = 0.6, font = \bfseries] at (1.75, 2.5) {-1};
        \node[black, scale = 0.6, font = \bfseries] at (1.75, 3) {1};
        \node[red, scale = 0.6, font = \bfseries] at (1.75, 3.5) {-1};
        \node[red, scale = 0.6, font = \bfseries] at (1.75, 4) {-1};
        \node[red, scale = 0.6, font = \bfseries] at (1.75, 4.5) {-1};
        \node[black, scale = 0.6, font = \bfseries] at (1.75, 5) {1};
        \node[red, scale = 0.6, font = \bfseries] at (1.75, 5.5) {-1};
        \node[red, scale = 0.6, font = \bfseries] at (1.75, 6) {-1};
        
        % 4. Spalte
        \node[red, scale = 0.6, font = \bfseries] at (2.25, 2) {-1};
        \node[red, scale = 0.6, font = \bfseries] at (2.25, 2.5) {-1};
        \node[red, scale = 0.6, font = \bfseries] at (2.25, 3) {-1};
        \node[black, scale = 0.6, font = \bfseries] at (2.25, 3.5) {1};
        \node[red, scale = 0.6, font = \bfseries] at (2.25, 4) {-1};
        \node[black, scale = 0.6, font = \bfseries] at (2.25, 4.5) {1};
        \node[red, scale = 0.6, font = \bfseries] at (2.25, 5) {-1};
        \node[red, scale = 0.6, font = \bfseries] at (2.25, 5.5) {-1};
        \node[red, scale = 0.6, font = \bfseries] at (2.25, 6) {-1};
        
        % 5. Spalte
        \node[red, scale = 0.6, font = \bfseries] at (2.75, 2) {-1};
        \node[red, scale = 0.6, font = \bfseries] at (2.75, 2.5) {-1};
        \node[red, scale = 0.6, font = \bfseries] at (2.75, 3) {-1};
        \node[red, scale = 0.6, font = \bfseries] at (2.75, 3.5) {-1};
        \node[black, scale = 0.6, font = \bfseries] at (2.75, 4) {1};
        \node[red, scale = 0.6, font = \bfseries] at (2.75, 4.5) {-1};
        \node[red, scale = 0.6, font = \bfseries] at (2.75, 5) {-1};
        \node[red, scale = 0.6, font = \bfseries] at (2.75, 5.5) {-1};
        \node[red, scale = 0.6, font = \bfseries] at (2.75, 6) {-1};
        
        % 6. Spalte
        \node[red, scale = 0.6, font = \bfseries] at (3.25, 2) {-1};
        \node[red, scale = 0.6, font = \bfseries] at (3.25, 2.5) {-1};
        \node[red, scale = 0.6, font = \bfseries] at (3.25, 3) {-1};
        \node[black, scale = 0.6, font = \bfseries] at (3.25, 3.5) {1};
        \node[red, scale = 0.6, font = \bfseries] at (3.25, 4) {-1};
        \node[black, scale = 0.6, font = \bfseries] at (3.25, 4.5) {1};
        \node[red, scale = 0.6, font = \bfseries] at (3.25, 5) {-1};
        \node[red, scale = 0.6, font = \bfseries] at (3.25, 5.5) {-1};
        \node[red, scale = 0.6, font = \bfseries] at (3.25, 6) {-1};
        
        % 7. Spalte
        \node[red, scale = 0.6, font = \bfseries] at (3.75, 2) {-1};
        \node[red, scale = 0.6, font = \bfseries] at (3.75, 2.5) {-1};
        \node[black, scale = 0.6, font = \bfseries] at (3.75, 3) {1};
        \node[red, scale = 0.6, font = \bfseries] at (3.75, 3.5) {-1};
        \node[red, scale = 0.6, font = \bfseries] at (3.75, 4) {-1};
        \node[red, scale = 0.6, font = \bfseries] at (3.75, 4.5) {-1};
        \node[black, scale = 0.6, font = \bfseries] at (3.75, 5) {1};
        \node[red, scale = 0.6, font = \bfseries] at (3.75, 5.5) {-1};
        \node[red, scale = 0.6, font = \bfseries] at (3.75, 6) {-1};
        
        % 8. Spalte
        \node[red, scale = 0.6, font = \bfseries] at (4.25, 2) {-1};
        \node[black, scale = 0.6, font = \bfseries] at (4.25, 2.5) {1};
        \node[red, scale = 0.6, font = \bfseries] at (4.25, 3) {-1};
        \node[red, scale = 0.6, font = \bfseries] at (4.25, 3.5) {-1};
        \node[red, scale = 0.6, font = \bfseries] at (4.25, 4) {-1};
        \node[red, scale = 0.6, font = \bfseries] at (4.25, 4.5) {-1};
        \node[red, scale = 0.6, font = \bfseries] at (4.25, 5) {-1};
        \node[black, scale = 0.6, font = \bfseries] at (4.25, 5.5) {1};
        \node[red, scale = 0.6, font = \bfseries] at (4.25, 6) {-1};
        
        % 9. Spalte
        \node[red, scale = 0.6, font = \bfseries] at (4.75, 2) {-1};
        \node[red, scale = 0.6, font = \bfseries] at (4.75, 2.5) {-1};
        \node[red, scale = 0.6, font = \bfseries] at (4.75, 3) {-1};
        \node[red, scale = 0.6, font = \bfseries] at (4.75, 3.5) {-1};
        \node[red, scale = 0.6, font = \bfseries] at (4.75, 4) {-1};
        \node[red, scale = 0.6, font = \bfseries] at (4.75, 4.5) {-1};
        \node[red, scale = 0.6, font = \bfseries] at (4.75, 5) {-1};
        \node[red, scale = 0.6, font = \bfseries] at (4.75, 5.5) {-1};
        \node[red, scale = 0.6, font = \bfseries] at (4.75, 6) {-1};
        
    % Zahlen positionieren im Eingangsbild 2
        % 1. Spalte
        \node[red, scale = 0.6, font = \bfseries] at (8.25, 2) {-1};
        \node[red, scale = 0.6, font = \bfseries] at (8.25, 2.5) {-1};
        \node[red, scale = 0.6, font = \bfseries] at (8.25, 3) {-1};
        \node[red, scale = 0.6, font = \bfseries] at (8.25, 3.5) {-1};
        \node[red, scale = 0.6, font = \bfseries] at (8.25, 4) {-1};
        \node[red, scale = 0.6, font = \bfseries] at (8.25, 4.5) {-1};
        \node[red, scale = 0.6, font = \bfseries] at (8.25, 5) {-1};
        \node[red, scale = 0.6, font = \bfseries] at (8.25, 5.5) {-1};
        \node[red, scale = 0.6, font = \bfseries] at (8.25, 6) {-1};
        
        % 2. Spalte
        \node[red, scale = 0.6, font = \bfseries] at (8.75, 2) {-1};
        \node[red, scale = 0.6, font = \bfseries] at (8.75, 2.5) {-1};
        \node[red, scale = 0.6, font = \bfseries] at (8.75, 3) {-1};
        \node[red, scale = 0.6, font = \bfseries] at (8.75, 3.5) {-1};
        \node[red, scale = 0.6, font = \bfseries] at (8.75, 4) {-1};
        \node[red, scale = 0.6, font = \bfseries] at (8.75, 4.5) {-1};
        \node[black, scale = 0.6, font = \bfseries] at (8.75, 5) {1};
        \node[red, scale = 0.6, font = \bfseries] at (8.75, 5.5) {1};
        \node[red, scale = 0.6, font = \bfseries] at (8.75, 6) {-1};
        
        % 3. Spalte
        \node[red, scale = 0.6, font = \bfseries] at (9.25, 2) {-1};
        \node[black, scale = 0.6, font = \bfseries] at (9.25, 2.5) {1};
        \node[red, scale = 0.6, font = \bfseries] at (9.25, 3) {-1};
        \node[red, scale = 0.6, font = \bfseries] at (9.25, 3.5) {-1};
        \node[red, scale = 0.6, font = \bfseries] at (9.25, 4) {-1};
        \node[black, scale = 0.6, font = \bfseries] at (9.25, 4.5) {1};
        \node[red, scale = 0.6, font = \bfseries] at (9.25, 5) {-1};
        \node[red, scale = 0.6, font = \bfseries] at (9.25, 5.5) {-1};
        \node[red, scale = 0.6, font = \bfseries] at (9.25, 6) {-1};
        
        % 4. Spalte
        \node[red, scale = 0.6, font = \bfseries] at (9.75, 2) {-1};
        \node[red, scale = 0.6, font = \bfseries] at (9.75, 2.5) {-1};
        \node[black, scale = 0.6, font = \bfseries] at (9.75, 3) {1};
        \node[black, scale = 0.6, font = \bfseries] at (9.75, 3.5) {1};
        \node[red, scale = 0.6, font = \bfseries] at (9.75, 4) {-1};
        \node[black, scale = 0.6, font = \bfseries] at (9.75, 4.5) {1};
        \node[red, scale = 0.6, font = \bfseries] at (9.75, 5) {-1};
        \node[red, scale = 0.6, font = \bfseries] at (9.75, 5.5) {-1};
        \node[red, scale = 0.6, font = \bfseries] at (9.75, 6) {-1};
        
        % 5. Spalte
        \node[red, scale = 0.6, font = \bfseries] at (10.25, 2) {-1};
        \node[red, scale = 0.6, font = \bfseries] at (10.25, 2.5) {-1};
        \node[red, scale = 0.6, font = \bfseries] at (10.25, 3) {-1};
        \node[red, scale = 0.6, font = \bfseries] at (10.25, 3.5) {-1};
        \node[black, scale = 0.6, font = \bfseries] at (10.25, 4) {1};
        \node[red, scale = 0.6, font = \bfseries] at (10.25, 4.5) {-1};
        \node[red, scale = 0.6, font = \bfseries] at (10.25, 5) {-1};
        \node[red, scale = 0.6, font = \bfseries] at (10.25, 5.5) {-1};
        \node[red, scale = 0.6, font = \bfseries] at (10.25, 6) {-1};
        
        % 6. Spalte
        \node[red, scale = 0.6, font = \bfseries] at (10.75, 2) {-1};
        \node[red, scale = 0.6, font = \bfseries] at (10.75, 2.5) {-1};
        \node[red, scale = 0.6, font = \bfseries] at (10.75, 3) {-1};
        \node[black, scale = 0.6, font = \bfseries] at (10.75, 3.5) {1};
        \node[red, scale = 0.6, font = \bfseries] at (10.75, 4) {-1};
        \node[black, scale = 0.6, font = \bfseries] at (10.75, 4.5) {1};
        \node[black, scale = 0.6, font = \bfseries] at (10.75, 5) {1};
        \node[red, scale = 0.6, font = \bfseries] at (10.75, 5.5) {-1};
        \node[red, scale = 0.6, font = \bfseries] at (10.75, 6) {-1};
        
        % 7. Spalte
        \node[red, scale = 0.6, font = \bfseries] at (11.25, 2) {-1};
        \node[red, scale = 0.6, font = \bfseries] at (11.25, 2.5) {-1};
        \node[red, scale = 0.6, font = \bfseries] at (11.25, 3) {-1};
        \node[black, scale = 0.6, font = \bfseries] at (11.25, 3.5) {1};
        \node[red, scale = 0.6, font = \bfseries] at (11.25, 4) {-1};
        \node[red, scale = 0.6, font = \bfseries] at (11.25, 4.5) {-1};
        \node[red, scale = 0.6, font = \bfseries] at (11.25, 5) {-1};
        \node[black, scale = 0.6, font = \bfseries] at (11.25, 5.5) {1};
        \node[red, scale = 0.6, font = \bfseries] at (11.25, 6) {-1};
        
        % 8. Spalte
        \node[red, scale = 0.6, font = \bfseries] at (11.75, 2) {-1};
        \node[red, scale = 0.6, font = \bfseries] at (11.75, 2.5) {-1};
        \node[black, scale = 0.6, font = \bfseries] at (11.75, 3) {1};
        \node[red, scale = 0.6, font = \bfseries] at (11.75, 3.5) {-1};
        \node[red, scale = 0.6, font = \bfseries] at (11.75, 4) {-1};
        \node[red, scale = 0.6, font = \bfseries] at (11.75, 4.5) {-1};
        \node[red, scale = 0.6, font = \bfseries] at (11.75, 5) {-1};
        \node[red, scale = 0.6, font = \bfseries] at (11.75, 5.5) {-1};
        \node[red, scale = 0.6, font = \bfseries] at (11.75, 6) {-1};
        
        % 9. Spalte
        \node[red, scale = 0.6, font = \bfseries] at (12.25, 2) {-1};
        \node[red, scale = 0.6, font = \bfseries] at (12.25, 2.5) {-1};
        \node[red, scale = 0.6, font = \bfseries] at (12.25, 3) {-1};
        \node[red, scale = 0.6, font = \bfseries] at (12.25, 3.5) {-1};
        \node[red, scale = 0.6, font = \bfseries] at (12.25, 4) {-1};
        \node[red, scale = 0.6, font = \bfseries] at (12.25, 4.5) {-1};
        \node[red, scale = 0.6, font = \bfseries] at (12.25, 5) {-1};
        \node[red, scale = 0.6, font = \bfseries] at (12.25, 5.5) {-1};
        \node[red, scale = 0.6, font = \bfseries] at (12.25, 6) {-1};
    
    % Umrandungen
        \draw[green, very thick] (1, 4.75) rectangle (2, 5.75);
        \draw[green, very thick] (8.5, 4.25) rectangle (9.5, 5.25);
        \draw[violet, very thick] (1, 2.25) rectangle (2, 3.25);
        \draw[violet, very thick] (9, 2.25) rectangle (10, 3.25);
        \draw[blue, very thick] (2, 3.25) rectangle (3.5, 4.75);
        \draw[blue, very thick] (9.5, 3.25) rectangle (11, 4.75);
    
    % Gleichheitszeichen
        \draw[black] (6.3, 2.5) -- (6.8, 2.5);
        \draw[black] (6.3, 2.75) -- (6.8, 2.75);
        \draw[black] (6.3, 3.75) -- (6.8, 3.75);
        \draw[black] (6.3, 4) -- (6.8, 4);
        \draw[black] (6.3, 5.25) -- (6.8, 5.25);
        \draw[black] (6.3, 5.5) -- (6.8, 5.5);
        
    % Gestrichelte Linien
        \draw[green, dashed] (2, 4.75) -- (6.1, 5.375);
        \draw[green, dashed] (2, 5.75) -- (6.1, 5.375);
        \draw[green, dashed] (7, 5.375) -- (8.5, 4.25);
        \draw[green, dashed] (7, 5.375) -- (8.5, 5.25);
        \draw[blue, dashed] (3.5, 4.75) -- (6.1, 3.875);
        \draw[blue, dashed] (3.5, 3.25) -- (6.1, 3.875);
        \draw[blue, dashed] (7, 3.875) -- (9.5, 4.75);
        \draw[blue, dashed] (7, 3.875) -- (9.5, 3.25);
        \draw[violet, dashed] (2, 3.25) -- (6.1, 2.625);
        \draw[violet, dashed] (2, 2.25) -- (6.1, 2.625);
        \draw[violet, dashed] (7, 2.625) -- (9, 3.25);
        \draw[violet, dashed] (7, 2.625) -- (9, 2.25);
    
    % Bezeichnungen
        \draw[decorate , decoration = {brace, amplitude = 10pt, mirror}] (0.5,1.5) -- ++(4.5, 0) node[black, midway, xshift = 0cm, yshift = -0.6cm] {\footnotesize $Eingangsbild\:1$};
        \draw[decorate , decoration = {brace, amplitude = 10pt, mirror}] (8,1.5) -- ++(4.5, 0) node[black, midway, xshift = 0cm, yshift = -0.6cm] {\footnotesize $Eingangsbild\:2$};
        
    \end{tikzpicture}
    }
    \caption[Unterschiedliche Eingangsbilder mit ähnlichen Merkmalen]{Unterschiedliche Eingangsbilder mit ähnlichen Merkmalen \cite{Wuertz2019}}
    \label{fig:Bild2.2}
\end{figure}

\subsubsection{Convolution} \label{sec: convolution}
Bei der Faltung werden Filter (Feature) mit \textbf{mathematischen Operationen} auf Teilbereiche des Eingangsbildes angewendet. Das Ergebnis der Faltung ist eine \textbf{Feature-Map} \cite{Wuertz2019}. Zur Veranschaulichung dient \autoref{fig:Bild2.3}. Der Filter hat beispielhaft eine Größe von 3x3px und wird auf die Teilbereiche im Abstand von 1px angewandt (s. grüne und blaue Markierung im Eingangsbild). Die mathematische Operation gleicht der Multiplikation der einzelnen Werte der Bildpixeln mit den Filterpixeln und der anschließenden \textbf{Mittelwertbildung}. Das Ergebnis wird an die jeweilige markierte Stelle in der Feature-Map übernommen \cite{Wuertz2019}.\\
Grüne Markierung:
\begin{align*}
    \begin{split}
    x &= \frac{(-1)\cdot1+(-1)\cdot(-1)+(-1)\cdot(-1)+(-1)\cdot(-1)+1\cdot1+(-1)\cdot(-1)}{9    }\\\
        & + \frac{(-1)\cdot(-1)+(-1)\cdot(-1)+1\cdot1}{9}
    \end{split} \\
    x &\approx 0.77
\end{align*}

Blaue Markierung:
\begin{align*}
    \begin{split}
    x &= \frac{(-1)\cdot1+1\cdot(-1)+(-1)\cdot(-1)+(-1)\cdot(-1)+(-1)\cdot1+1\cdot(-1)}{9    }\\\
        & + \frac{(-1)\cdot(-1)+(-1)\cdot(-1)+(-1)\cdot1}{9}
    \end{split} \\
    x &\approx -0.11
\end{align*}

\begin{figure}[H]
    \centering
    \scalebox{0.9}
    {
    \begin{tikzpicture}[framed][domain=0:0]
    % Größe der Bildumgebung
        \draw[black] (0, 0) rectangle (15, 8);
    
    % Größe des Eingangsbilds
        \draw[black] (0.5, 1.75) rectangle (5.0, 6.25);
    
    % Vertikale Linien des Eingangsbilds
        \draw[black, thin] (1.0, 1.75) -- ++(0, 4.5);
        \draw[black, thin] (1.5, 1.75) -- ++(0, 4.5);
        \draw[black, thin] (2.0, 1.75) -- ++(0, 4.5);
        \draw[black, thin] (2.5, 1.75) -- ++(0, 4.5);
        \draw[black, thin] (3.0, 1.75) -- ++(0, 4.5);
        \draw[black, thin] (3.5, 1.75) -- ++(0, 4.5);
        \draw[black, thin] (4.0, 1.75) -- ++(0, 4.5);
        \draw[black, thin] (4.5, 1.75) -- ++(0, 4.5);
    
    % Horizontale Linien des Eingangsbilds
        \draw[black, thin] (0.5, 2.25) -- ++(4.5, 0);
        \draw[black, thin] (0.5, 2.75) -- ++(4.5, 0);
        \draw[black, thin] (0.5, 3.25) -- ++(4.5, 0);
        \draw[black, thin] (0.5, 3.75) -- ++(4.5, 0);
        \draw[black, thin] (0.5, 4.25) -- ++(4.5, 0);
        \draw[black, thin] (0.5, 4.75) -- ++(4.5, 0);
        \draw[black, thin] (0.5, 5.25) -- ++(4.5, 0);
        \draw[black, thin] (0.5, 5.75) -- ++(4.5, 0);
    
    % Größe der Feature-Map
        \draw[black] (11, 2.25) rectangle (14.5, 5.75);
        
    % Horizontale Linien der Feature-Map
        \draw[black, thin] (11, 2.75) -- ++(3.5, 0);
        \draw[black, thin] (11, 3.25) -- ++(3.5, 0);
        \draw[black, thin] (11, 3.75) -- ++(3.5, 0);
        \draw[black, thin] (11, 4.25) -- ++(3.5, 0);
        \draw[black, thin] (11, 4.75) -- ++(3.5, 0);
        \draw[black, thin] (11, 5.25) -- ++(3.5, 0);
        
    % Vertikale Linien der Feature-Map
        \draw[black, thin] (11.5, 2.25) -- ++(0, 3.5);
        \draw[black, thin] (12.0, 2.25) -- ++(0, 3.5);
        \draw[black, thin] (12.5, 2.25) -- ++(0, 3.5);
        \draw[black, thin] (13.0, 2.25) -- ++(0, 3.5);
        \draw[black, thin] (13.5, 2.25) -- ++(0, 3.5);
        \draw[black, thin] (14.0, 2.25) -- ++(0, 3.5);
        
    % Größe des Filters
        \draw[green, very thick] (7.25, 3.25) rectangle (8.75, 4.75);
        
    % Vertikale Linien des Filters
        \draw[black, thin] (7.75, 3.25) -- ++(0, 1.5);
        \draw[black, thin] (8.25, 3.25) -- ++(0, 1.5);
        
    % Horizontale Linien des Filters
        \draw[black, thin] (7.25, 3.75) -- ++(1.5, 0);
        \draw[black, thin] (7.25, 4.25) -- ++(1.5, 0);
        
    % Zahlen positionieren im Eingangsbild
        % 1. Spalte
        \node[red, scale = 0.6, font = \bfseries] at (0.75, 2) {-1};
        \node[red, scale = 0.6, font = \bfseries] at (0.75, 2.5) {-1};
        \node[red, scale = 0.6, font = \bfseries] at (0.75, 3) {-1};
        \node[red, scale = 0.6, font = \bfseries] at (0.75, 3.5) {-1};
        \node[red, scale = 0.6, font = \bfseries] at (0.75, 4) {-1};
        \node[red, scale = 0.6, font = \bfseries] at (0.75, 4.5) {-1};
        \node[red, scale = 0.6, font = \bfseries] at (0.75, 5) {-1};
        \node[red, scale = 0.6, font = \bfseries] at (0.75, 5.5) {-1};
        \node[red, scale = 0.6, font = \bfseries] at (0.75, 6) {-1};
        
        % 2. Spalte
        \node[red, scale = 0.6, font = \bfseries] at (1.25, 2) {-1};
        \node[black, scale = 0.6, font = \bfseries] at (1.25, 2.5) {1};
        \node[red, scale = 0.6, font = \bfseries] at (1.25, 3) {-1};
        \node[red, scale = 0.6, font = \bfseries] at (1.25, 3.5) {-1};
        \node[red, scale = 0.6, font = \bfseries] at (1.25, 4) {-1};
        \node[red, scale = 0.6, font = \bfseries] at (1.25, 4.5) {-1};
        \node[red, scale = 0.6, font = \bfseries] at (1.25, 5) {-1};
        \node[black, scale = 0.6, font = \bfseries] at (1.25, 5.5) {1};
        \node[red, scale = 0.6, font = \bfseries] at (1.25, 6) {-1};
        
        % 3. Spalte
        \node[red, scale = 0.6, font = \bfseries] at (1.75, 2) {-1};
        \node[red, scale = 0.6, font = \bfseries] at (1.75, 2.5) {-1};
        \node[black, scale = 0.6, font = \bfseries] at (1.75, 3) {1};
        \node[red, scale = 0.6, font = \bfseries] at (1.75, 3.5) {-1};
        \node[red, scale = 0.6, font = \bfseries] at (1.75, 4) {-1};
        \node[red, scale = 0.6, font = \bfseries] at (1.75, 4.5) {-1};
        \node[black, scale = 0.6, font = \bfseries] at (1.75, 5) {1};
        \node[red, scale = 0.6, font = \bfseries] at (1.75, 5.5) {-1};
        \node[red, scale = 0.6, font = \bfseries] at (1.75, 6) {-1};
        
        % 4. Spalte
        \node[red, scale = 0.6, font = \bfseries] at (2.25, 2) {-1};
        \node[red, scale = 0.6, font = \bfseries] at (2.25, 2.5) {-1};
        \node[red, scale = 0.6, font = \bfseries] at (2.25, 3) {-1};
        \node[black, scale = 0.6, font = \bfseries] at (2.25, 3.5) {1};
        \node[red, scale = 0.6, font = \bfseries] at (2.25, 4) {-1};
        \node[black, scale = 0.6, font = \bfseries] at (2.25, 4.5) {1};
        \node[red, scale = 0.6, font = \bfseries] at (2.25, 5) {-1};
        \node[red, scale = 0.6, font = \bfseries] at (2.25, 5.5) {-1};
        \node[red, scale = 0.6, font = \bfseries] at (2.25, 6) {-1};
        
        % 5. Spalte
        \node[red, scale = 0.6, font = \bfseries] at (2.75, 2) {-1};
        \node[red, scale = 0.6, font = \bfseries] at (2.75, 2.5) {-1};
        \node[red, scale = 0.6, font = \bfseries] at (2.75, 3) {-1};
        \node[red, scale = 0.6, font = \bfseries] at (2.75, 3.5) {-1};
        \node[black, scale = 0.6, font = \bfseries] at (2.75, 4) {1};
        \node[red, scale = 0.6, font = \bfseries] at (2.75, 4.5) {-1};
        \node[red, scale = 0.6, font = \bfseries] at (2.75, 5) {-1};
        \node[red, scale = 0.6, font = \bfseries] at (2.75, 5.5) {-1};
        \node[red, scale = 0.6, font = \bfseries] at (2.75, 6) {-1};
        
        % 6. Spalte
        \node[red, scale = 0.6, font = \bfseries] at (3.25, 2) {-1};
        \node[red, scale = 0.6, font = \bfseries] at (3.25, 2.5) {-1};
        \node[red, scale = 0.6, font = \bfseries] at (3.25, 3) {-1};
        \node[black, scale = 0.6, font = \bfseries] at (3.25, 3.5) {1};
        \node[red, scale = 0.6, font = \bfseries] at (3.25, 4) {-1};
        \node[black, scale = 0.6, font = \bfseries] at (3.25, 4.5) {1};
        \node[red, scale = 0.6, font = \bfseries] at (3.25, 5) {-1};
        \node[red, scale = 0.6, font = \bfseries] at (3.25, 5.5) {-1};
        \node[red, scale = 0.6, font = \bfseries] at (3.25, 6) {-1};
        
        % 7. Spalte
        \node[red, scale = 0.6, font = \bfseries] at (3.75, 2) {-1};
        \node[red, scale = 0.6, font = \bfseries] at (3.75, 2.5) {-1};
        \node[black, scale = 0.6, font = \bfseries] at (3.75, 3) {1};
        \node[red, scale = 0.6, font = \bfseries] at (3.75, 3.5) {-1};
        \node[red, scale = 0.6, font = \bfseries] at (3.75, 4) {-1};
        \node[red, scale = 0.6, font = \bfseries] at (3.75, 4.5) {-1};
        \node[black, scale = 0.6, font = \bfseries] at (3.75, 5) {1};
        \node[red, scale = 0.6, font = \bfseries] at (3.75, 5.5) {-1};
        \node[red, scale = 0.6, font = \bfseries] at (3.75, 6) {-1};
        
        % 8. Spalte
        \node[red, scale = 0.6, font = \bfseries] at (4.25, 2) {-1};
        \node[black, scale = 0.6, font = \bfseries] at (4.25, 2.5) {1};
        \node[red, scale = 0.6, font = \bfseries] at (4.25, 3) {-1};
        \node[red, scale = 0.6, font = \bfseries] at (4.25, 3.5) {-1};
        \node[red, scale = 0.6, font = \bfseries] at (4.25, 4) {-1};
        \node[red, scale = 0.6, font = \bfseries] at (4.25, 4.5) {-1};
        \node[red, scale = 0.6, font = \bfseries] at (4.25, 5) {-1};
        \node[black, scale = 0.6, font = \bfseries] at (4.25, 5.5) {1};
        \node[red, scale = 0.6, font = \bfseries] at (4.25, 6) {-1};
        
        % 9. Spalte
        \node[red, scale = 0.6, font = \bfseries] at (4.75, 2) {-1};
        \node[red, scale = 0.6, font = \bfseries] at (4.75, 2.5) {-1};
        \node[red, scale = 0.6, font = \bfseries] at (4.75, 3) {-1};
        \node[red, scale = 0.6, font = \bfseries] at (4.75, 3.5) {-1};
        \node[red, scale = 0.6, font = \bfseries] at (4.75, 4) {-1};
        \node[red, scale = 0.6, font = \bfseries] at (4.75, 4.5) {-1};
        \node[red, scale = 0.6, font = \bfseries] at (4.75, 5) {-1};
        \node[red, scale = 0.6, font = \bfseries] at (4.75, 5.5) {-1};
        \node[red, scale = 0.6, font = \bfseries] at (4.75, 6) {-1};
        
    % Zahlen positionieren des Filters
        % 1. Spalte
        \node[red, scale = 0.6, font = \bfseries] at (7.5, 3.5) {-1};
        \node[red, scale = 0.6, font = \bfseries] at (7.5, 4.0) {-1};
        \node[black, scale = 0.6, font = \bfseries] at (7.5, 4.5) {1};
        
        % 2. Spalte
        \node[red, scale = 0.6, font = \bfseries] at (8, 3.5) {-1};
        \node[black, scale = 0.6, font = \bfseries] at (8, 4.0) {1};
        \node[red, scale = 0.6, font = \bfseries] at (8, 4.5) {-1};
        
        % 3. Spalte
        \node[black, scale = 0.6, font = \bfseries] at (8.5, 3.5) {1};
        \node[red, scale = 0.6, font = \bfseries] at (8.5, 4.0) {-1};
        \node[red, scale = 0.6, font = \bfseries] at (8.5, 4.5) {-1};
    
    % Zahlen positionieren der Feature-Map
        % 1. Spalte
        \node[orange, scale = 0.55, font = \bfseries] at (11.25, 2.5) { .33};
        \node[orange, scale = 0.55, font = \bfseries] at (11.25, 3.0) {-.11};
        \node[orange, scale = 0.55, font = \bfseries] at (11.25, 3.5) { .55};
        \node[orange, scale = 0.55, font = \bfseries] at (11.25, 4.0) { .33};
        \node[orange, scale = 0.55, font = \bfseries] at (11.25, 4.5) { .11};
        \node[orange, scale = 0.55, font = \bfseries] at (11.25, 5.0) {-.11};
        \node[cyan, scale = 0.55, font = \bfseries] at (11.25, 5.5) { .77};
        
        % 2. Spalte
        \node[orange, scale = 0.55, font = \bfseries] at (11.75, 2.5) {-.11};
        \node[orange, scale = 0.55, font = \bfseries] at (11.75, 3.0) { .11};
        \node[orange, scale = 0.55, font = \bfseries] at (11.75, 3.5) {-.11};
        \node[orange, scale = 0.55, font = \bfseries] at (11.75, 4.0) { .33};
        \node[orange, scale = 0.55, font = \bfseries] at (11.75, 4.5) {-.11};
        \node[cyan, scale = 0.55, font = \bfseries] at (11.75, 5.0) {1.00};
        \node[orange, scale = 0.55, font = \bfseries] at (11.75, 5.5) {-.11};
        
        % 3. Spalte
        \node[orange, scale = 0.55, font = \bfseries] at (12.25, 2.5) { .55};
        \node[orange, scale = 0.55, font = \bfseries] at (12.25, 3.0) {-.11};
        \node[orange, scale = 0.55, font = \bfseries] at (12.25, 3.5) { .11};
        \node[orange, scale = 0.55, font = \bfseries] at (12.25, 4.0) {-.33};
        \node[cyan, scale = 0.55, font = \bfseries] at (12.25, 4.5) {1.00};
        \node[orange, scale = 0.55, font = \bfseries] at (12.25, 5.0) {-.11};
        \node[orange, scale = 0.55, font = \bfseries] at (12.25, 5.5) { .11};
        
        % 4. Spalte
        \node[orange, scale = 0.55, font = \bfseries] at (12.75, 2.5) { .33};
        \node[orange, scale = 0.55, font = \bfseries] at (12.75, 3.0) { .33};
        \node[orange, scale = 0.55, font = \bfseries] at (12.75, 3.5) {-.33};
        \node[cyan, scale = 0.55, font = \bfseries] at (12.75, 4.0) { .55};
        \node[orange, scale = 0.55, font = \bfseries] at (12.75, 4.5) {-.33};
        \node[orange, scale = 0.55, font = \bfseries] at (12.75, 5.0) { .33};
        \node[orange, scale = 0.55, font = \bfseries] at (12.75, 5.5) { .33};
        
        % 5. Spalte
        \node[orange, scale = 0.55, font = \bfseries] at (13.25, 2.5) { .11};
        \node[orange, scale = 0.55, font = \bfseries] at (13.25, 3.0) {-.11};
        \node[cyan, scale = 0.55, font = \bfseries] at (13.25, 3.5) {1.00};
        \node[orange, scale = 0.55, font = \bfseries] at (13.25, 4.0) {-.33};
        \node[orange, scale = 0.55, font = \bfseries] at (13.25, 4.5) { .11};
        \node[orange, scale = 0.55, font = \bfseries] at (13.25, 5.0) {-.11};
        \node[orange, scale = 0.55, font = \bfseries] at (13.25, 5.5) { .55};
        
        % 6. Spalte
        \node[orange, scale = 0.55, font = \bfseries] at (13.75, 2.5) {-.11};
        \node[cyan, scale = 0.55, font = \bfseries] at (13.75, 3.0) {1.00};
        \node[orange, scale = 0.55, font = \bfseries] at (13.75, 3.5) {-.11};
        \node[orange, scale = 0.55, font = \bfseries] at (13.75, 4.0) { .33};
        \node[orange, scale = 0.55, font = \bfseries] at (13.75, 4.5) {-.11};
        \node[orange, scale = 0.55, font = \bfseries] at (13.75, 5.0) {.11};
        \node[orange, scale = 0.55, font = \bfseries] at (13.75, 5.5) {-.11};
        
        % 7. Spalte
        \node[cyan, scale = 0.55, font = \bfseries] at (14.25, 2.5) { .77};
        \node[orange, scale = 0.55, font = \bfseries] at (14.25, 3.0) {-.11};
        \node[orange, scale = 0.55, font = \bfseries] at (14.25, 3.5) { .11};
        \node[orange, scale = 0.55, font = \bfseries] at (14.25, 4.0) { .33};
        \node[orange, scale = 0.55, font = \bfseries] at (14.25, 4.5) { .55};
        \node[orange, scale = 0.55, font = \bfseries] at (14.25, 5.0) {-.11};
        \node[orange, scale = 0.55, font = \bfseries] at (14.25, 5.5) { .33};
        
    % Faltungssymbol und Gleichheitszeichen
        \draw[] (6.125, 4) circle (0.25);
        \draw[] (6.125, 4) -- ++(45:0.25cm);
        \draw[] (6.125, 4) -- ++(135:0.25cm);
        \draw[] (6.125, 4) -- ++(-45:0.25cm);
        \draw[] (6.125, 4) -- ++(-135:0.25cm);
        \draw[] (9.625, 4.1) -- ++(0.5, 0);
        \draw[] (9.625, 3.9) -- ++(0.5, 0);
    
    % Bezeichnungen
        \draw[decorate , decoration = {brace, amplitude = 10pt, mirror}] (0.5,1.5) -- ++(4.5, 0) node[black, midway, xshift = 0cm, yshift = -0.6cm] {\footnotesize $Eingangsbild$};
        \draw[decorate , decoration = {brace, amplitude = 10pt, mirror}] (7.25,1.5) -- ++(1.5, 0) node[black, midway, xshift = 0cm, yshift = -0.6cm] {\footnotesize $Feature$};
        \draw[decorate , decoration = {brace, amplitude = 10pt, mirror}] (11,1.5) -- ++(3.5, 0) node[black, midway, xshift = 0cm, yshift = -0.6cm] {\footnotesize $Feature-Map$};
    
    % Umrandungen
        \draw[green, very thick] (0.5, 4.75) rectangle (2, 6.25);
        \draw[green, very thick] (11, 5.25) rectangle  (11.5, 5.75);
        \draw[blue, very thick] (0.5, 4.25) rectangle (2, 5.75);
        \draw[blue, very thick] (11, 4.75) rectangle (11.5, 5.25);
        \draw[blue, very thick] (7.125, 3.125) rectangle (8.875, 4.875);
        
    % Markierungen
        \draw[green] (1.25, 6.5) -- (1.25, 6.75) -- (11.25, 6.75) -- (11.25, 6);
        \draw[green] (8, 5) -- (8, 6.75);
        \draw[blue] (0.35, 5) -- (0.25, 5) -- (0.25, 7.25) -- (10.5, 7.25) -- (10.5, 5) -- (10.75, 5);
        \draw[blue] (9, 4) -- (9.3, 4) -- (9.3, 5) -- (10.5, 5);
    
    \end{tikzpicture}
    }
    \caption[Anwendung der Faltung mithilfe eines Filters]{Anwendung der Faltung mithilfe eines Filters \cite{Wuertz2019}}
    \label{fig:Bild2.3}
\end{figure}

Anhand eines Filters kann keine genaue Auswertung vorgenommen werden. Folglich ist die Anwendung unterschiedlicher Filter notwendig. Jeder Filter erzeugt eine andere Feature-Map, in welcher Auszüge des Originalbilds mit unterschiedlicher Gewichtung erkennbar sind (\autoref{fig:Bild2.4}).

\begin{figure}[H]
    \centering
    \scalebox{0.9}
    {
    \begin{tikzpicture}[framed][domain=0:0]
    % Größe der Bildumgebung
        \draw[black] (0, 0) rectangle (15, 7.5);
    
    % 1. Filter
        % Größe des Filters
        \draw[black] (1.5, 5.5) rectangle (3, 7);
        
        % Vertikale Linien des Filters
        \draw[black, thin] (2, 5.5) -- ++(0, 1.5);
        \draw[black, thin] (2.5, 5.5) -- ++(0, 1.5);
        
        % Horizontale Linien des Filters
        \draw[black, thin] (1.5, 6.0) -- ++(1.5, 0);
        \draw[black, thin] (1.5, 6.5) -- ++(1.5, 0);
    
        % Zahlen positionieren des Filters
        % 1. Spalte
        \node[red, scale = 0.6, font = \bfseries] at (1.75, 5.75) {-1};
        \node[red, scale = 0.6, font = \bfseries] at (1.75, 6.25) {-1};
        \node[black, scale = 0.6, font = \bfseries] at (1.75, 6.75) {1};
        
        % 2. Spalte
        \node[red, scale = 0.6, font = \bfseries] at (2.25, 5.75) {-1};
        \node[black, scale = 0.6, font = \bfseries] at (2.25, 6.25) {1};
        \node[red, scale = 0.6, font = \bfseries] at (2.25, 6.75) {-1};
        
        % 3. Spalte
        \node[black, scale = 0.6, font = \bfseries] at (2.75, 5.75) {1};
        \node[red, scale = 0.6, font = \bfseries] at (2.75, 6.25) {-1};
        \node[red, scale = 0.6, font = \bfseries] at (2.75, 6.75) {-1};
    
    % 1. Feature-Map
        % Größe der Feature-Map
        \draw[black] (0.5, 0.5) rectangle (4, 4.0);
        
         % Horizontale Linien der Feature-Map
        \draw[black, thin] (0.5, 1.0) -- ++(3.5, 0);
        \draw[black, thin] (0.5, 1.5) -- ++(3.5, 0);
        \draw[black, thin] (0.5, 2.0) -- ++(3.5, 0);
        \draw[black, thin] (0.5, 2.5) -- ++(3.5, 0);
        \draw[black, thin] (0.5, 3.0) -- ++(3.5, 0);
        \draw[black, thin] (0.5, 3.5) -- ++(3.5, 0);
        
        % Vertikale Linien der Feature-Map
        \draw[black, thin] (1.0, 0.5) -- ++(0, 3.5);
        \draw[black, thin] (1.5, 0.5) -- ++(0, 3.5);
        \draw[black, thin] (2.0, 0.5) -- ++(0, 3.5);
        \draw[black, thin] (2.5, 0.5) -- ++(0, 3.5);
        \draw[black, thin] (3.0, 0.5) -- ++(0, 3.5);
        \draw[black, thin] (3.5, 0.5) -- ++(0, 3.5);
        
        % Zahlen positionieren der Feature-Map
        % 1. Spalte
        \node[orange, scale = 0.55, font = \bfseries] at (0.75, 0.75) { .33};
        \node[orange, scale = 0.55, font = \bfseries] at (0.75, 1.25) {-.11};
        \node[orange, scale = 0.55, font = \bfseries] at (0.75, 1.75) { .55};
        \node[orange, scale = 0.55, font = \bfseries] at (0.75, 2.25) { .33};
        \node[orange, scale = 0.55, font = \bfseries] at (0.75, 2.75) { .11};
        \node[orange, scale = 0.55, font = \bfseries] at (0.75, 3.25) {-.11};
        \node[cyan, scale = 0.55, font = \bfseries] at (0.75, 3.75) { .77};
        
        % 2. Spalte
        \node[orange, scale = 0.55, font = \bfseries] at (1.25, 0.75) {-.11};
        \node[orange, scale = 0.55, font = \bfseries] at (1.25, 1.25) { .11};
        \node[orange, scale = 0.55, font = \bfseries] at (1.25, 1.75) {-.11};
        \node[orange, scale = 0.55, font = \bfseries] at (1.25, 2.25) { .33};
        \node[orange, scale = 0.55, font = \bfseries] at (1.25, 2.75) {-.11};
        \node[cyan, scale = 0.55, font = \bfseries] at (1.25, 3.25) {1.00};
        \node[orange, scale = 0.55, font = \bfseries] at (1.25, 3.75) {-.11};
        
        % 3. Spalte
        \node[orange, scale = 0.55, font = \bfseries] at (1.75, 0.75) { .55};
        \node[orange, scale = 0.55, font = \bfseries] at (1.75, 1.25) {-.11};
        \node[orange, scale = 0.55, font = \bfseries] at (1.75, 1.75) { .11};
        \node[orange, scale = 0.55, font = \bfseries] at (1.75, 2.25) {-.33};
        \node[cyan, scale = 0.55, font = \bfseries] at (1.75, 2.75) {1.00};
        \node[orange, scale = 0.55, font = \bfseries] at (1.75, 3.25) {-.11};
        \node[orange, scale = 0.55, font = \bfseries] at (1.75, 3.75) { .11};
        
        % 4. Spalte
        \node[orange, scale = 0.55, font = \bfseries] at (2.25, 0.75) { .33};
        \node[orange, scale = 0.55, font = \bfseries] at (2.25, 1.25) { .33};
        \node[orange, scale = 0.55, font = \bfseries] at (2.25, 1.75) {-.33};
        \node[cyan, scale = 0.55, font = \bfseries] at (2.25, 2.25) { .55};
        \node[orange, scale = 0.55, font = \bfseries] at (2.25, 2.75) {-.33};
        \node[orange, scale = 0.55, font = \bfseries] at (2.25, 3.25) { .33};
        \node[orange, scale = 0.55, font = \bfseries] at (2.25, 3.75) { .33};
        
        % 5. Spalte
        \node[orange, scale = 0.55, font = \bfseries] at (2.75, 0.75) { .11};
        \node[orange, scale = 0.55, font = \bfseries] at (2.75, 1.25) {-.11};
        \node[cyan, scale = 0.55, font = \bfseries] at (2.75, 1.75) {1.00};
        \node[orange, scale = 0.55, font = \bfseries] at (2.75, 2.25) {-.33};
        \node[orange, scale = 0.55, font = \bfseries] at (2.75, 2.75) { .11};
        \node[orange, scale = 0.55, font = \bfseries] at (2.75, 3.25) {-.11};
        \node[orange, scale = 0.55, font = \bfseries] at (2.75, 3.75) { .55};
        
        % 6. Spalte
        \node[orange, scale = 0.55, font = \bfseries] at (3.25, 0.75) {-.11};
        \node[cyan, scale = 0.55, font = \bfseries] at (3.25, 1.25) {1.00};
        \node[orange, scale = 0.55, font = \bfseries] at (3.25, 1.75) {-.11};
        \node[orange, scale = 0.55, font = \bfseries] at (3.25, 2.25) { .33};
        \node[orange, scale = 0.55, font = \bfseries] at (3.25, 2.75) {-.11};
        \node[orange, scale = 0.55, font = \bfseries] at (3.25, 3.25) {.11};
        \node[orange, scale = 0.55, font = \bfseries] at (3.25, 3.75) {-.11};
        
        % 7. Spalte
        \node[cyan, scale = 0.55, font = \bfseries] at (3.75, 0.75) { .77};
        \node[orange, scale = 0.55, font = \bfseries] at (3.75, 1.25) {-.11};
        \node[orange, scale = 0.55, font = \bfseries] at (3.75, 1.75) { .11};
        \node[orange, scale = 0.55, font = \bfseries] at (3.75, 2.25) { .33};
        \node[orange, scale = 0.55, font = \bfseries] at (3.75, 2.75) { .55};
        \node[orange, scale = 0.55, font = \bfseries] at (3.75, 3.25) {-.11};
        \node[orange, scale = 0.55, font = \bfseries] at (3.75, 3.75) { .33};
    
    % 2. Filter
        % Größe des Filters
        \draw[black] (6.75, 5.5) rectangle (8.25, 7);
        
        % Vertikale Linien des Filters
        \draw[black, thin] (7.25, 5.5) -- ++(0, 1.5);
        \draw[black, thin] (7.75, 5.5) -- ++(0, 1.5);
        
        % Horizontale Linien des Filters
        \draw[black, thin] (6.75, 6.0) -- ++(1.5, 0);
        \draw[black, thin] (6.75, 6.5) -- ++(1.5, 0);
    
        % Zahlen positionieren des Filters
        % 1. Spalte
        \node[black, scale = 0.6, font = \bfseries] at (7.0, 5.75) {1};
        \node[red, scale = 0.6, font = \bfseries] at (7.0, 6.25) {-1};
        \node[red, scale = 0.6, font = \bfseries] at (7.0, 6.75) {-1};
        
        % 2. Spalte
        \node[red, scale = 0.6, font = \bfseries] at (7.5, 5.75) {-1};
        \node[black, scale = 0.6, font = \bfseries] at (7.5, 6.25) {1};
        \node[red, scale = 0.6, font = \bfseries] at (7.5, 6.75) {-1};
        
        % 3. Spalte
        \node[red, scale = 0.6, font = \bfseries] at (8.0, 5.75) {-1};
        \node[red, scale = 0.6, font = \bfseries] at (8.0, 6.25) {-1};
        \node[black, scale = 0.6, font = \bfseries] at (8.0, 6.75) {1};
    
    % 2. Feature-Map
        % Größe der Feature-Map
        \draw[black] (5.75, 0.5) rectangle (9.25, 4.0);
        
         % Horizontale Linien der Feature-Map
        \draw[black, thin] (5.75, 1.0) -- ++(3.5, 0);
        \draw[black, thin] (5.75, 1.5) -- ++(3.5, 0);
        \draw[black, thin] (5.75, 2.0) -- ++(3.5, 0);
        \draw[black, thin] (5.75, 2.5) -- ++(3.5, 0);
        \draw[black, thin] (5.75, 3.0) -- ++(3.5, 0);
        \draw[black, thin] (5.75, 3.5) -- ++(3.5, 0);
        
        % Vertikale Linien der Feature-Map
        \draw[black, thin] (6.25, 0.5) -- ++(0, 3.5);
        \draw[black, thin] (6.75, 0.5) -- ++(0, 3.5);
        \draw[black, thin] (7.25, 0.5) -- ++(0, 3.5);
        \draw[black, thin] (7.75, 0.5) -- ++(0, 3.5);
        \draw[black, thin] (8.25, 0.5) -- ++(0, 3.5);
        \draw[black, thin] (8.75, 0.5) -- ++(0, 3.5);
        
        % Zahlen positionieren der Feature-Map
        % 1. Spalte
        \node[cyan, scale = 0.55, font = \bfseries] at (6.0, 0.75) { .77};
        \node[orange, scale = 0.55, font = \bfseries] at (6.0, 1.25) {-.11};
        \node[orange, scale = 0.55, font = \bfseries] at (6.0, 1.75) { .11};
        \node[orange, scale = 0.55, font = \bfseries] at (6.0, 2.25) { .33};
        \node[orange, scale = 0.55, font = \bfseries] at (6.0, 2.75) { .55};
        \node[orange, scale = 0.55, font = \bfseries] at (6.0, 3.25) {-.11};
        \node[orange, scale = 0.55, font = \bfseries] at (6.0, 3.75) { .33};
        
        % 2. Spalte
        \node[orange, scale = 0.55, font = \bfseries] at (6.5, 0.75) {-.11};
        \node[cyan, scale = 0.55, font = \bfseries] at (6.5, 1.25) {1.00};
        \node[orange, scale = 0.55, font = \bfseries] at (6.5, 1.75) {-.11};
        \node[orange, scale = 0.55, font = \bfseries] at (6.5, 2.25) { .33};
        \node[orange, scale = 0.55, font = \bfseries] at (6.5, 2.75) {-.11};
        \node[orange, scale = 0.55, font = \bfseries] at (6.5, 3.25) { .11};
        \node[orange, scale = 0.55, font = \bfseries] at (6.5, 3.75) {-.11};
        
        % 3. Spalte
        \node[orange, scale = 0.55, font = \bfseries] at (7.0, 0.75) { .11};
        \node[orange, scale = 0.55, font = \bfseries] at (7.0, 1.25) {-.11};
        \node[cyan, scale = 0.55, font = \bfseries] at (7.0, 1.75) {1.00};
        \node[orange, scale = 0.55, font = \bfseries] at (7.0, 2.25) {-.33};
        \node[orange, scale = 0.55, font = \bfseries] at (7.0, 2.75) { .11};
        \node[orange, scale = 0.55, font = \bfseries] at (7.0, 3.25) {-.11};
        \node[orange, scale = 0.55, font = \bfseries] at (7.0, 3.75) { .55};
        
        % 4. Spalte
        \node[orange, scale = 0.55, font = \bfseries] at (7.5, 0.75) { .33};
        \node[orange, scale = 0.55, font = \bfseries] at (7.5, 1.25) { .33};
        \node[orange, scale = 0.55, font = \bfseries] at (7.5, 1.75) {-.33};
        \node[cyan, scale = 0.55, font = \bfseries] at (7.5, 2.25) { .55};
        \node[orange, scale = 0.55, font = \bfseries] at (7.5, 2.75) {-.33};
        \node[orange, scale = 0.55, font = \bfseries] at (7.5, 3.25) { .33};
        \node[orange, scale = 0.55, font = \bfseries] at (7.5, 3.75) { .33};
        
        % 5. Spalte
        \node[orange, scale = 0.55, font = \bfseries] at (8.0, 0.75) { .55};
        \node[orange, scale = 0.55, font = \bfseries] at (8.0, 1.25) {-.11};
        \node[orange, scale = 0.55, font = \bfseries] at (8.0, 1.75) { .11};
        \node[orange, scale = 0.55, font = \bfseries] at (8.0, 2.25) {-.33};
        \node[cyan, scale = 0.55, font = \bfseries] at (8.0, 2.75) {1.00};
        \node[orange, scale = 0.55, font = \bfseries] at (8.0, 3.25) {-.11};
        \node[orange, scale = 0.55, font = \bfseries] at (8.0, 3.75) { .11};
        
        % 6. Spalte
        \node[orange, scale = 0.55, font = \bfseries] at (8.5, 0.75) {-.11};
        \node[orange, scale = 0.55, font = \bfseries] at (8.5, 1.25) { .11};
        \node[orange, scale = 0.55, font = \bfseries] at (8.5, 1.75) {-.11};
        \node[orange, scale = 0.55, font = \bfseries] at (8.5, 2.25) { .33};
        \node[orange, scale = 0.55, font = \bfseries] at (8.5, 2.75) {-.11};
        \node[cyan, scale = 0.55, font = \bfseries] at (8.5, 3.25) {1.00};
        \node[orange, scale = 0.55, font = \bfseries] at (8.5, 3.75) {-.11};
        
        % 7. Spalte
        \node[orange, scale = 0.55, font = \bfseries] at (9.0, 0.75) { .33};
        \node[orange, scale = 0.55, font = \bfseries] at (9.0, 1.25) {-.11};
        \node[orange, scale = 0.55, font = \bfseries] at (9.0, 1.75) { .55};
        \node[orange, scale = 0.55, font = \bfseries] at (9.0, 2.25) { .33};
        \node[orange, scale = 0.55, font = \bfseries] at (9.0, 2.75) { .11};
        \node[orange, scale = 0.55, font = \bfseries] at (9.0, 3.25) {-.11};
        \node[cyan, scale = 0.55, font = \bfseries] at (9.0, 3.75) { .77};
    
    % 3. Filter
        % Größe des Filters
        \draw[black] (12.0, 5.5) rectangle (13.5, 7);
        
        % Vertikale Linien des Filters
        \draw[black, thin] (12.5, 5.5) -- ++(0, 1.5);
        \draw[black, thin] (13.0, 5.5) -- ++(0, 1.5);
        
        % Horizontale Linien des Filters
        \draw[black, thin] (12.0, 6.0) -- ++(1.5, 0);
        \draw[black, thin] (12.0, 6.5) -- ++(1.5, 0);
    
        % Zahlen positionieren des Filters
        % 1. Spalte
        \node[black, scale = 0.6, font = \bfseries] at (12.25, 5.75) {1};
        \node[red, scale = 0.6, font = \bfseries] at (12.25, 6.25) {-1};
        \node[black, scale = 0.6, font = \bfseries] at (12.25, 6.75) {1};
        
        % 2. Spalte
        \node[red, scale = 0.6, font = \bfseries] at (12.75, 5.75) {-1};
        \node[black, scale = 0.6, font = \bfseries] at (12.75, 6.25) {1};
        \node[red, scale = 0.6, font = \bfseries] at (12.75, 6.75) {-1};
        
        % 3. Spalte
        \node[black, scale = 0.6, font = \bfseries] at (13.25, 5.75) {1};
        \node[red, scale = 0.6, font = \bfseries] at (13.25, 6.25) {-1};
        \node[black, scale = 0.6, font = \bfseries] at (13.25, 6.75) {1};
    
    % 3. Feature-Map
        % Größe der Feature-Map
        \draw[black] (11, 0.5) rectangle (14.5, 4.0);
        
         % Horizontale Linien der Feature-Map
        \draw[black, thin] (11, 1.0) -- ++(3.5, 0);
        \draw[black, thin] (11, 1.5) -- ++(3.5, 0);
        \draw[black, thin] (11, 2.0) -- ++(3.5, 0);
        \draw[black, thin] (11, 2.5) -- ++(3.5, 0);
        \draw[black, thin] (11, 3.0) -- ++(3.5, 0);
        \draw[black, thin] (11, 3.5) -- ++(3.5, 0);
        
        % Vertikale Linien der Feature-Map
        \draw[black, thin] (11.5, 0.5) -- ++(0, 3.5);
        \draw[black, thin] (12.0, 0.5) -- ++(0, 3.5);
        \draw[black, thin] (12.5, 0.5) -- ++(0, 3.5);
        \draw[black, thin] (13.0, 0.5) -- ++(0, 3.5);
        \draw[black, thin] (13.5, 0.5) -- ++(0, 3.5);
        \draw[black, thin] (14.0, 0.5) -- ++(0, 3.5);
        
        % Zahlen positionieren der Feature-Map
        % 1. Spalte
        \node[cyan, scale = 0.55, font = \bfseries] at (11.25, 0.75) { .33};
        \node[orange, scale = 0.55, font = \bfseries] at (11.25, 1.25) {-.55};
        \node[orange, scale = 0.55, font = \bfseries] at (11.25, 1.75) { .11};
        \node[orange, scale = 0.55, font = \bfseries] at (11.25, 2.25) {-.11};
        \node[orange, scale = 0.55, font = \bfseries] at (11.25, 2.75) { .11};
        \node[orange, scale = 0.55, font = \bfseries] at (11.25, 3.25) {-.55};
        \node[cyan, scale = 0.55, font = \bfseries] at (11.25, 3.75) { .33};
        
        % 2. Spalte
        \node[orange, scale = 0.55, font = \bfseries] at (11.75, 0.75) {-.55};
        \node[cyan, scale = 0.55, font = \bfseries] at (11.75, 1.25) { .55};
        \node[orange, scale = 0.55, font = \bfseries] at (11.75, 1.75) {-.55};
        \node[orange, scale = 0.55, font = \bfseries] at (11.75, 2.25) { .33};
        \node[orange, scale = 0.55, font = \bfseries] at (11.75, 2.75) {-.55};
        \node[cyan, scale = 0.55, font = \bfseries] at (11.75, 3.25) {0.55};
        \node[orange, scale = 0.55, font = \bfseries] at (11.75, 3.75) {-.55};
        
        % 3. Spalte
        \node[orange, scale = 0.55, font = \bfseries] at (12.25, 0.75) { .11};
        \node[orange, scale = 0.55, font = \bfseries] at (12.25, 1.25) {-.55};
        \node[cyan, scale = 0.55, font = \bfseries] at (12.25, 1.75) { .55};
        \node[orange, scale = 0.55, font = \bfseries] at (12.25, 2.25) {-.77};
        \node[cyan, scale = 0.55, font = \bfseries] at (12.25, 2.75) { .55};
        \node[orange, scale = 0.55, font = \bfseries] at (12.25, 3.25) {-.55};
        \node[orange, scale = 0.55, font = \bfseries] at (12.25, 3.75) { .11};
        
        % 4. Spalte
        \node[orange, scale = 0.55, font = \bfseries] at (12.75, 0.75) {-.11};
        \node[orange, scale = 0.55, font = \bfseries] at (12.75, 1.25) { .33};
        \node[orange, scale = 0.55, font = \bfseries] at (12.75, 1.75) {-.77};
        \node[cyan, scale = 0.55, font = \bfseries] at (12.75, 2.25) {1.00};
        \node[orange, scale = 0.55, font = \bfseries] at (12.75, 2.75) {-.77};
        \node[orange, scale = 0.55, font = \bfseries] at (12.75, 3.25) { .33};
        \node[orange, scale = 0.55, font = \bfseries] at (12.75, 3.75) {-.11};
        
        % 5. Spalte
        \node[orange, scale = 0.55, font = \bfseries] at (13.25, 0.75) { .11};
        \node[orange, scale = 0.55, font = \bfseries] at (13.25, 1.25) {-.55};
        \node[cyan, scale = 0.55, font = \bfseries] at (13.25, 1.75) { .55};
        \node[orange, scale = 0.55, font = \bfseries] at (13.25, 2.25) {-.77};
        \node[cyan, scale = 0.55, font = \bfseries] at (13.25, 2.75) { .55};
        \node[orange, scale = 0.55, font = \bfseries] at (13.25, 3.25) {-.55};
        \node[orange, scale = 0.55, font = \bfseries] at (13.25, 3.75) { .11};
        
        % 6. Spalte
        \node[orange, scale = 0.55, font = \bfseries] at (13.75, 0.75) {-.55};
        \node[cyan, scale = 0.55, font = \bfseries] at (13.75, 1.25) { .55};
        \node[orange, scale = 0.55, font = \bfseries] at (13.75, 1.75) {-.55};
        \node[orange, scale = 0.55, font = \bfseries] at (13.75, 2.25) { .33};
        \node[orange, scale = 0.55, font = \bfseries] at (13.75, 2.75) {-.55};
        \node[cyan, scale = 0.55, font = \bfseries] at (13.75, 3.25) {.55};
        \node[orange, scale = 0.55, font = \bfseries] at (13.75, 3.75) {-.55};
        
        % 7. Spalte
        \node[cyan, scale = 0.55, font = \bfseries] at (14.25, 0.75) { .33};
        \node[orange, scale = 0.55, font = \bfseries] at (14.25, 1.25) {-.55};
        \node[orange, scale = 0.55, font = \bfseries] at (14.25, 1.75) { .11};
        \node[orange, scale = 0.55, font = \bfseries] at (14.25, 2.25) {-.11};
        \node[orange, scale = 0.55, font = \bfseries] at (14.25, 2.75) { .11};
        \node[orange, scale = 0.55, font = \bfseries] at (14.25, 3.25) {-.55};
        \node[cyan, scale = 0.55, font = \bfseries] at (14.25, 3.75) { .33};
    
    % Markierungen
        \draw[green, thick] (0.5, 4) -- (1.5, 5.5);
        \draw[green, thick] (4, 4) -- (3, 5.5);
        \draw[green, thick] (5.75, 4) -- (6.75, 5.5);
        \draw[green, thick] (9.25, 4) -- (8.25, 5.5);
        \draw[green, thick] (11, 4) -- (12, 5.5);
        \draw[green, thick] (14.5, 4) -- (13.5, 5.5);
    
    \end{tikzpicture}
    }
    \caption[Filtertypen und Feature-Maps]{Filtertypen und zugehörige Feature-Maps \cite{Wuertz2019}}
    \label{fig:Bild2.4}
\end{figure}

    

\subsubsection{Pooling} \label{sec: pooling}
Beim Maximum Pooling wird in Teilbereichen der Feature-Map nach einem maximalen Wert gesucht. Die Werte werden anschließend zu einer verkleinerten Feature-Map zusammengesetzt. Dies hat den Vorteil, dass Speicherplatz gespart und die höchsten Gewichtungen der Merkmale extrahiert werden können \cite{Wuertz2019}. In \autoref{fig:Bild2.5} wird exemplarisch eine Fenstergröße von 2x2px mit einer Schrittweite von 2px gewählt.

\begin{figure}[H]
    \centering
    \scalebox{0.9}
    {
    \begin{tikzpicture}[framed][domain=0:0]
    % Größe der Bildumgebung
        \draw[black] (0, 0) rectangle (10, 6);
    
    % Größe der Feature-Map
        \draw[black] (0.5, 1.75) rectangle (4, 5.25);
        
    % Horizontale Linien der Feature-Map
        \draw[black, thin] (0.5, 2.25) -- ++(3.5, 0);
        \draw[black, thin] (0.5, 2.75) -- ++(3.5, 0);
        \draw[black, thin] (0.5, 3.25) -- ++(3.5, 0);
        \draw[black, thin] (0.5, 3.75) -- ++(3.5, 0);
        \draw[black, thin] (0.5, 4.25) -- ++(3.5, 0);
        \draw[black, thin] (0.5, 4.75) -- ++(3.5, 0);
        
    % Vertikale Linien der Feature-Map
        \draw[black, thin] (1, 1.75) -- ++(0, 3.5);
        \draw[black, thin] (1.5, 1.75) -- ++(0, 3.5);
        \draw[black, thin] (2, 1.75) -- ++(0, 3.5);
        \draw[black, thin] (2.5, 1.75) -- ++(0, 3.5);
        \draw[black, thin] (3, 1.75) -- ++(0, 3.5);
        \draw[black, thin] (3.5, 1.75) -- ++(0, 3.5);

    % Zahlen positionieren der Feature-Map
        % 1. Spalte
        \node[orange, scale = 0.55, font = \bfseries] at (0.75, 2) { .33};
        \node[orange, scale = 0.55, font = \bfseries] at (0.75, 2.5) {-.11};
        \node[orange, scale = 0.55, font = \bfseries] at (0.75, 3) { .55};
        \node[orange, scale = 0.55, font = \bfseries] at (0.75, 3.5) { .33};
        \node[orange, scale = 0.55, font = \bfseries] at (0.75, 4) { .11};
        \node[orange, scale = 0.55, font = \bfseries] at (0.75, 4.5) {-.11};
        \node[cyan, scale = 0.55, font = \bfseries] at (0.75, 5) { .77};
        
        % 2. Spalte
        \node[orange, scale = 0.55, font = \bfseries] at (1.25, 2) {-.11};
        \node[orange, scale = 0.55, font = \bfseries] at (1.25, 2.5) { .11};
        \node[orange, scale = 0.55, font = \bfseries] at (1.25, 3) {-.11};
        \node[orange, scale = 0.55, font = \bfseries] at (1.25, 3.5) { .33};
        \node[orange, scale = 0.55, font = \bfseries] at (1.25, 4) {-.11};
        \node[cyan, scale = 0.55, font = \bfseries] at (1.25, 4.5) {1.00};
        \node[orange, scale = 0.55, font = \bfseries] at (1.25, 5) {-.11};
        
        % 3. Spalte
        \node[orange, scale = 0.55, font = \bfseries] at (1.75, 2) { .55};
        \node[orange, scale = 0.55, font = \bfseries] at (1.75, 2.5) {-.11};
        \node[orange, scale = 0.55, font = \bfseries] at (1.75, 3) { .11};
        \node[orange, scale = 0.55, font = \bfseries] at (1.75, 3.5) {-.33};
        \node[cyan, scale = 0.55, font = \bfseries] at (1.75, 4) {1.00};
        \node[orange, scale = 0.55, font = \bfseries] at (1.75, 4.5) {-.11};
        \node[orange, scale = 0.55, font = \bfseries] at (1.75, 5) { .11};
        
        % 4. Spalte
        \node[orange, scale = 0.55, font = \bfseries] at (2.25, 2) { .33};
        \node[orange, scale = 0.55, font = \bfseries] at (2.25, 2.5) { .33};
        \node[orange, scale = 0.55, font = \bfseries] at (2.25, 3) {-.33};
        \node[cyan, scale = 0.55, font = \bfseries] at (2.25, 3.5) { .55};
        \node[orange, scale = 0.55, font = \bfseries] at (2.25, 4) {-.33};
        \node[orange, scale = 0.55, font = \bfseries] at (2.25, 4.5) { .33};
        \node[orange, scale = 0.55, font = \bfseries] at (2.25, 5) { .33};
        
        % 5. Spalte
        \node[orange, scale = 0.55, font = \bfseries] at (2.75, 2) { .11};
        \node[orange, scale = 0.55, font = \bfseries] at (2.75, 2.5) {-.11};
        \node[cyan, scale = 0.55, font = \bfseries] at (2.75, 3) {1.00};
        \node[orange, scale = 0.55, font = \bfseries] at (2.75, 3.5) {-.33};
        \node[orange, scale = 0.55, font = \bfseries] at (2.75, 4) { .11};
        \node[orange, scale = 0.55, font = \bfseries] at (2.75, 4.5) {-.11};
        \node[orange, scale = 0.55, font = \bfseries] at (2.75, 5) { .55};
        
        % 6. Spalte
        \node[orange, scale = 0.55, font = \bfseries] at (3.25, 2) {-.11};
        \node[cyan, scale = 0.55, font = \bfseries] at (3.25, 2.5) {1.00};
        \node[orange, scale = 0.55, font = \bfseries] at (3.25, 3) {-.11};
        \node[orange, scale = 0.55, font = \bfseries] at (3.25, 3.5) { .33};
        \node[orange, scale = 0.55, font = \bfseries] at (3.25, 4) {-.11};
        \node[orange, scale = 0.55, font = \bfseries] at (3.25, 4.5) {.11};
        \node[orange, scale = 0.55, font = \bfseries] at (3.25, 5) {-.11};
        
        % 7. Spalte
        \node[cyan, scale = 0.55, font = \bfseries] at (3.75, 2) { .77};
        \node[orange, scale = 0.55, font = \bfseries] at (3.75, 2.5) {-.11};
        \node[orange, scale = 0.55, font = \bfseries] at (3.75, 3) { .11};
        \node[orange, scale = 0.55, font = \bfseries] at (3.75, 3.5) { .33};
        \node[orange, scale = 0.55, font = \bfseries] at (3.75, 4) { .55};
        \node[orange, scale = 0.55, font = \bfseries] at (3.75, 4.5) {-.11};
        \node[orange, scale = 0.55, font = \bfseries] at (3.75, 5) { .33};
    
    % Feature Map nach Pooling
        % Größe der Feature-Map
        \draw[black] (7, 2.25) rectangle (9, 4.25);
         
        % Horizontale Linien
        \draw[black, thin] (7, 2.75) -- ++(2, 0);
        \draw[black, thin] (7, 3.25) -- ++(2, 0);
        \draw[black, thin] (7, 3.75) -- ++(2, 0);
        
        % Vertikale Linien
        \draw[black, thin] (7.5, 2.25) -- ++(0, 2);
        \draw[black, thin] (8, 2.25) -- ++(0, 2);
        \draw[black, thin] (8.5, 2.25) -- ++(0, 2);
        
        % Zahlen positionieren
        % 1. Spalte
        \node[orange, scale = 0.55, font = \bfseries] at (7.25, 2.5) { .33};
        \node[orange, scale = 0.55, font = \bfseries] at (7.25, 3) { .55};
        \node[orange, scale = 0.55, font = \bfseries] at (7.25, 3.5) { .33};
        \node[cyan, scale = 0.55, font = \bfseries] at (7.25, 4) {1.00};
        
        % 2. Spalte
        \node[orange, scale = 0.55, font = \bfseries] at (7.75, 2.5) { .55};
        \node[orange, scale = 0.55, font = \bfseries] at (7.75, 3) { .33};
        \node[cyan, scale = 0.55, font = \bfseries] at (7.75, 3.5) {1.00};
        \node[orange, scale = 0.55, font = \bfseries] at (7.75, 4) { .33}; 
        
        % 3. Spalte
        \node[orange, scale = 0.55, font = \bfseries] at (8.25, 2.5) { .11};
        \node[cyan, scale = 0.55, font = \bfseries] at (8.25, 3) {1.00};
        \node[orange, scale = 0.55, font = \bfseries] at (8.25, 3.5) { .33};
        \node[orange, scale = 0.55, font = \bfseries] at (8.25, 4) { .55};
        
        % 4. Spalte
        \node[cyan, scale = 0.55, font = \bfseries] at (8.75, 2.5) { .77};
        \node[orange, scale = 0.55, font = \bfseries] at (8.75, 3) { .11};
        \node[orange, scale = 0.55, font = \bfseries] at (8.75, 3.5) { .55};
        \node[orange, scale = 0.55, font = \bfseries] at (8.75, 4) { .33};
         
    % Umrandungen
        \draw[green, very thick] (0.5, 4.25) rectangle (1.5, 5.25);
        \draw[blue, very thick] (0.5, 3.25) rectangle (1.5, 4.25);
        \draw[green, very thick] (7, 3.75) rectangle (7.5, 4.24);
        \draw[blue, very thick] (7, 3.25) rectangle (7.5, 3.75);
    
    % Weitere Markierungen
        \draw[black, dashed] (4, 5.25) -- (7, 4.25);
        \draw[black, dashed] (4, 1.75) -- (7, 2.25);
        
    % Bezeichnungen
        \draw[decorate , decoration = {brace, amplitude = 5pt, mirror}] (0.5,1.25) -- ++(3.5, 0) node[black, midway, xshift = 0cm, yshift = -0.6cm] {\footnotesize $vor\:max.\:Pooling$};
        \draw[decorate , decoration = {brace, amplitude = 5pt, mirror}] (7,1.25) -- ++(2, 0) node[black, midway, xshift = 0cm, yshift = -0.6cm] {\footnotesize $nach\:max.\:Pooling$};
    \end{tikzpicture}
    }
    \caption[Anwendung der Zusammenlegung (max. Pooling)]{Anwendung der Zusammenlegung (max. Pooling) \cite{Wuertz2019}}
    \label{fig:Bild2.5}
\end{figure}

\subsubsection{Rectification} \label{sec: rectification}
Im Rectification-Layer werden alle negativen Werte einer Feature-Map zu Null angenommen (\autoref{fig:Bild2.7}). Dieser Prozess reduziert den Rechenaufwand. Zur Anwendung kommt eine lineare Funktion (Rectified Linear Unit (ReLU)) (\autoref{fig:Bild2.6}), die Funktionswerte größer Null unverändert lässt \cite{Wuertz2019}.

\begin{figure}[H]
    \centering
    \scalebox{0.9}
    {
    \begin{tikzpicture}[framed][domain=0:0]
        % Umgebung
        \draw[very thin,color=black] (-0.1,-1.1);
        
        % Koordinatenursprung
        \node[black] at (-0.2, -0.3) {0};
        
        % Achsenpunkte
        \draw[black, very thin] (0, 1) -- (-0.1, 1);
        \draw[black, very thin] (0, 2) -- (-0.1, 2);
        
        \draw[black, very thin] (1, 0) -- (1, -0.1);
        \draw[black, very thin] (2, 0) -- (2, -0.1);
        \draw[black, very thin] (-1, 0) -- (-1, -0.1);
        
        \node[black] at (1, -0.4) {1};
        \node[black] at (2, -0.4) {2};
        \node[black] at (-1, -0.4) {-1};
        
        \node[black] at (-0.3, 1) {1};
        \node[black] at (-0.3, 2) {2};
        
        % X-Achse
        \draw[->] (-2,0) -- (3,0) node[right] {$x$};
        
        % Y-Achse
        \draw[->] (0,-1) -- (0,3) node[above] {$y$};
        
        % Funktion
        \draw[orange, very thick] (-2, 0) -- (0, 0) -- (2.5, 2.5);
        \draw[orange, very thick, dashed] (2.5, 2.5) -- (3, 3);
        
        % Weitere Markierungen
        \draw[black, very thin, dashed] (1, 0) -- (1, 1) -- (0, 1);
        \draw[black, very thin, dashed] (2, 0) -- (2, 2) -- (0, 2);
        
    \end{tikzpicture}
    }
    \caption[Rectified Linear Unit]{Rectified Linear Unit \cite{Wuertz2019}}
    \label{fig:Bild2.6}
\end{figure}

\begin{figure}[H]
    \centering
    \scalebox{0.9}
    {
    \begin{tikzpicture}[framed][domain=0:0]
    % Größe der Bildumgebung
        \draw[black] (0, 0) rectangle (10, 6);
    
    % Größe der Feature-Map
        \draw[black] (0.5, 1.75) rectangle (4, 5.25);
        
    % Horizontale Linien der Feature-Map
        \draw[black, thin] (0.5, 2.25) -- ++(3.5, 0);
        \draw[black, thin] (0.5, 2.75) -- ++(3.5, 0);
        \draw[black, thin] (0.5, 3.25) -- ++(3.5, 0);
        \draw[black, thin] (0.5, 3.75) -- ++(3.5, 0);
        \draw[black, thin] (0.5, 4.25) -- ++(3.5, 0);
        \draw[black, thin] (0.5, 4.75) -- ++(3.5, 0);
        
    % Vertikale Linien der Feature-Map
        \draw[black, thin] (1, 1.75) -- ++(0, 3.5);
        \draw[black, thin] (1.5, 1.75) -- ++(0, 3.5);
        \draw[black, thin] (2, 1.75) -- ++(0, 3.5);
        \draw[black, thin] (2.5, 1.75) -- ++(0, 3.5);
        \draw[black, thin] (3, 1.75) -- ++(0, 3.5);
        \draw[black, thin] (3.5, 1.75) -- ++(0, 3.5);

    % Zahlen positionieren der Feature-Map
        % 1. Spalte
        \node[orange, scale = 0.55, font = \bfseries] at (0.75, 2) { .33};
        \node[orange, scale = 0.55, font = \bfseries] at (0.75, 2.5) {-.11};
        \node[orange, scale = 0.55, font = \bfseries] at (0.75, 3) { .55};
        \node[orange, scale = 0.55, font = \bfseries] at (0.75, 3.5) { .33};
        \node[orange, scale = 0.55, font = \bfseries] at (0.75, 4) { .11};
        \node[orange, scale = 0.55, font = \bfseries] at (0.75, 4.5) {-.11};
        \node[cyan, scale = 0.55, font = \bfseries] at (0.75, 5) { .77};
        
        % 2. Spalte
        \node[orange, scale = 0.55, font = \bfseries] at (1.25, 2) {-.11};
        \node[orange, scale = 0.55, font = \bfseries] at (1.25, 2.5) { .11};
        \node[orange, scale = 0.55, font = \bfseries] at (1.25, 3) {-.11};
        \node[orange, scale = 0.55, font = \bfseries] at (1.25, 3.5) { .33};
        \node[orange, scale = 0.55, font = \bfseries] at (1.25, 4) {-.11};
        \node[cyan, scale = 0.55, font = \bfseries] at (1.25, 4.5) {1.00};
        \node[orange, scale = 0.55, font = \bfseries] at (1.25, 5) {-.11};
        
        % 3. Spalte
        \node[orange, scale = 0.55, font = \bfseries] at (1.75, 2) { .55};
        \node[orange, scale = 0.55, font = \bfseries] at (1.75, 2.5) {-.11};
        \node[orange, scale = 0.55, font = \bfseries] at (1.75, 3) { .11};
        \node[orange, scale = 0.55, font = \bfseries] at (1.75, 3.5) {-.33};
        \node[cyan, scale = 0.55, font = \bfseries] at (1.75, 4) {1.00};
        \node[orange, scale = 0.55, font = \bfseries] at (1.75, 4.5) {-.11};
        \node[orange, scale = 0.55, font = \bfseries] at (1.75, 5) { .11};
        
        % 4. Spalte
        \node[orange, scale = 0.55, font = \bfseries] at (2.25, 2) { .33};
        \node[orange, scale = 0.55, font = \bfseries] at (2.25, 2.5) { .33};
        \node[orange, scale = 0.55, font = \bfseries] at (2.25, 3) {-.33};
        \node[cyan, scale = 0.55, font = \bfseries] at (2.25, 3.5) { .55};
        \node[orange, scale = 0.55, font = \bfseries] at (2.25, 4) {-.33};
        \node[orange, scale = 0.55, font = \bfseries] at (2.25, 4.5) { .33};
        \node[orange, scale = 0.55, font = \bfseries] at (2.25, 5) { .33};
        
        % 5. Spalte
        \node[orange, scale = 0.55, font = \bfseries] at (2.75, 2) { .11};
        \node[orange, scale = 0.55, font = \bfseries] at (2.75, 2.5) {-.11};
        \node[cyan, scale = 0.55, font = \bfseries] at (2.75, 3) {1.00};
        \node[orange, scale = 0.55, font = \bfseries] at (2.75, 3.5) {-.33};
        \node[orange, scale = 0.55, font = \bfseries] at (2.75, 4) { .11};
        \node[orange, scale = 0.55, font = \bfseries] at (2.75, 4.5) {-.11};
        \node[orange, scale = 0.55, font = \bfseries] at (2.75, 5) { .55};
        
        % 6. Spalte
        \node[orange, scale = 0.55, font = \bfseries] at (3.25, 2) {-.11};
        \node[cyan, scale = 0.55, font = \bfseries] at (3.25, 2.5) {1.00};
        \node[orange, scale = 0.55, font = \bfseries] at (3.25, 3) {-.11};
        \node[orange, scale = 0.55, font = \bfseries] at (3.25, 3.5) { .33};
        \node[orange, scale = 0.55, font = \bfseries] at (3.25, 4) {-.11};
        \node[orange, scale = 0.55, font = \bfseries] at (3.25, 4.5) {.11};
        \node[orange, scale = 0.55, font = \bfseries] at (3.25, 5) {-.11};
        
        % 7. Spalte
        \node[cyan, scale = 0.55, font = \bfseries] at (3.75, 2) { .77};
        \node[orange, scale = 0.55, font = \bfseries] at (3.75, 2.5) {-.11};
        \node[orange, scale = 0.55, font = \bfseries] at (3.75, 3) { .11};
        \node[orange, scale = 0.55, font = \bfseries] at (3.75, 3.5) { .33};
        \node[orange, scale = 0.55, font = \bfseries] at (3.75, 4) { .55};
        \node[orange, scale = 0.55, font = \bfseries] at (3.75, 4.5) {-.11};
        \node[orange, scale = 0.55, font = \bfseries] at (3.75, 5) { .33};
    
    % Feature Map nach Pooling
        % Größe der Feature-Map
        \draw[black] (6, 1.75) rectangle (9.5, 5.25);
         
        % Horizontale Linien
        \draw[black, thin] (6, 2.25) -- ++(3.5, 0);
        \draw[black, thin] (6, 2.75) -- ++(3.5, 0);
        \draw[black, thin] (6, 3.25) -- ++(3.5, 0);
        \draw[black, thin] (6, 3.75) -- ++(3.5, 0);
        \draw[black, thin] (6, 4.25) -- ++(3.5, 0);
        \draw[black, thin] (6, 4.75) -- ++(3.5, 0);
        
        % Vertikale Linien
        \draw[black, thin] (6.5, 1.75) -- ++(0, 3.5);
        \draw[black, thin] (7, 1.75) -- ++(0, 3.5);
        \draw[black, thin] (7.5, 1.75) -- ++(0, 3.5);
        \draw[black, thin] (8, 1.75) -- ++(0, 3.5);
        \draw[black, thin] (8.5, 1.75) -- ++(0, 3.5);
        \draw[black, thin] (9, 1.75) -- ++(0, 3.5);
        
        % Zahlen positionieren der Feature-Map
        % 1. Spalte
        \node[orange, scale = 0.55, font = \bfseries] at (6.25, 2) { .33};
        \node[orange, scale = 0.55, font = \bfseries] at (6.25, 2.5) {0};
        \node[orange, scale = 0.55, font = \bfseries] at (6.25, 3) { .55};
        \node[orange, scale = 0.55, font = \bfseries] at (6.25, 3.5) { .33};
        \node[orange, scale = 0.55, font = \bfseries] at (6.25, 4) { .11};
        \node[orange, scale = 0.55, font = \bfseries] at (6.25, 4.5) {0};
        \node[cyan, scale = 0.55, font = \bfseries] at (6.25, 5) { .77};
        
        % 2. Spalte
        \node[orange, scale = 0.55, font = \bfseries] at (6.75, 2) {0};
        \node[orange, scale = 0.55, font = \bfseries] at (6.75, 2.5) { .11};
        \node[orange, scale = 0.55, font = \bfseries] at (6.75, 3) {0};
        \node[orange, scale = 0.55, font = \bfseries] at (6.75, 3.5) { .33};
        \node[orange, scale = 0.55, font = \bfseries] at (6.75, 4) {0};
        \node[cyan, scale = 0.55, font = \bfseries] at (6.75, 4.5) {1.00};
        \node[orange, scale = 0.55, font = \bfseries] at (6.75, 5) {0};
        
        % 3. Spalte
        \node[orange, scale = 0.55, font = \bfseries] at (7.25, 2) { .55};
        \node[orange, scale = 0.55, font = \bfseries] at (7.25, 2.5) {0};
        \node[orange, scale = 0.55, font = \bfseries] at (7.25, 3) { .11};
        \node[orange, scale = 0.55, font = \bfseries] at (7.25, 3.5) {0};
        \node[cyan, scale = 0.55, font = \bfseries] at (7.25, 4) {1.00};
        \node[orange, scale = 0.55, font = \bfseries] at (7.25, 4.5) {0};
        \node[orange, scale = 0.55, font = \bfseries] at (7.25, 5) { .11};
        
        % 4. Spalte
        \node[orange, scale = 0.55, font = \bfseries] at (7.75, 2) { .33};
        \node[orange, scale = 0.55, font = \bfseries] at (7.75, 2.5) { .33};
        \node[orange, scale = 0.55, font = \bfseries] at (7.75, 3) {0};
        \node[cyan, scale = 0.55, font = \bfseries] at (7.75, 3.5) { .55};
        \node[orange, scale = 0.55, font = \bfseries] at (7.75, 4) {0};
        \node[orange, scale = 0.55, font = \bfseries] at (7.75, 4.5) { .33};
        \node[orange, scale = 0.55, font = \bfseries] at (7.75, 5) { .33};
        
        % 5. Spalte
        \node[orange, scale = 0.55, font = \bfseries] at (8.25, 2) { .11};
        \node[orange, scale = 0.55, font = \bfseries] at (8.25, 2.5) {0};
        \node[cyan, scale = 0.55, font = \bfseries] at (8.25, 3) {1.00};
        \node[orange, scale = 0.55, font = \bfseries] at (8.25, 3.5) {0};
        \node[orange, scale = 0.55, font = \bfseries] at (8.25, 4) { .11};
        \node[orange, scale = 0.55, font = \bfseries] at (8.25, 4.5) {0};
        \node[orange, scale = 0.55, font = \bfseries] at (8.25, 5) { .55};
        
        % 6. Spalte
        \node[orange, scale = 0.55, font = \bfseries] at (8.75, 2) {0};
        \node[cyan, scale = 0.55, font = \bfseries] at (8.75, 2.5) {1.00};
        \node[orange, scale = 0.55, font = \bfseries] at (8.75, 3) {0};
        \node[orange, scale = 0.55, font = \bfseries] at (8.75, 3.5) { .33};
        \node[orange, scale = 0.55, font = \bfseries] at (8.75, 4) {0};
        \node[orange, scale = 0.55, font = \bfseries] at (8.75, 4.5) {.11};
        \node[orange, scale = 0.55, font = \bfseries] at (8.75, 5) {0};
        
        % 7. Spalte
        \node[cyan, scale = 0.55, font = \bfseries] at (9.25, 2) { .77};
        \node[orange, scale = 0.55, font = \bfseries] at (9.25, 2.5) {0};
        \node[orange, scale = 0.55, font = \bfseries] at (9.25, 3) { .11};
        \node[orange, scale = 0.55, font = \bfseries] at (9.25, 3.5) { .33};
        \node[orange, scale = 0.55, font = \bfseries] at (9.25, 4) { .55};
        \node[orange, scale = 0.55, font = \bfseries] at (9.25, 4.5) {0};
        \node[orange, scale = 0.55, font = \bfseries] at (9.25, 5) { .33};
 
    % Bezeichnungen
        \draw[decorate , decoration = {brace, amplitude = 5pt, mirror}] (0.5,1.25) -- ++(3.5, 0) node[black, midway, xshift = 0cm, yshift = -0.6cm] {\footnotesize $vor\:Gleichrichtung$};
        \draw[decorate , decoration = {brace, amplitude = 5pt, mirror}] (6,1.25) -- ++(3.5, 0) node[black, midway, xshift = 0cm, yshift = -0.6cm] {\footnotesize $nach\:Gleichrichtung$};
    
    \end{tikzpicture}
    }
    \caption[Anwendung der Gleichrichtung (Rectification)]{Anwendung der Gleichrichtung (Rectification) \cite{Wuertz2019}}
    \label{fig:Bild2.7}
\end{figure}

\subsubsection{Fully-Connected} \label{sec: fully-connected}
Im Fully-Connected-Layer werden die Feature-Maps der vorangegangenen Schicht ausgewertet, um die Wahrscheinlichkeiten der Klassenzugehörigkeiten zu ermitteln \cite{Wuertz2019}. In \autoref{fig:Bild2.8} werden exemplarisch die fünf größten und fünf kleinsten Gewichtungen gemittelt. Das jeweilige Ergebnis spiegelt die Wahrscheinlichkeit wieder, dass das Eingangsbild der Klasse \glqq X\grqq\:oder \glqq kein X\grqq\:zugeordnet werden kann \cite{Wuertz2019}.\\
\newline
Wahrscheinlichkeit Klasse \glqq X\grqq:
\begin{align*}
    x &= \frac{0.90+0.87+0.96+0.89+0.94}{5}\\
    x &\approx 0.912\:(91.2\%)
\end{align*}

Wahrscheinlichkeit Klasse \glqq kein X\grqq:
\begin{align*}
    x &= \frac{0.45+0.23+0.63+0.44+0.53}{5}\\
    x &\approx 0.456\:(45.6\%)
\end{align*}

\begin{figure}[H]
    \centering
    \scalebox{0.9}
    {
    \begin{tikzpicture}[framed][domain=0:0]
    % Größe der Bildumgebung
        \draw[black] (0, 0) rectangle (14, 7);
    
    % Größe der Feature-Maps
        \draw[black] (2.5, 1) rectangle (3.5, 2);
        \draw[black] (2.5, 3) rectangle (3.5, 4);
        \draw[black] (2.5, 5) rectangle (3.5, 6);
    
    % Vertikale Linien der Feature-Maps
        \draw[black] (3, 1) -- ++(0, 1);
        \draw[black] (3, 3) -- ++(0, 1);
        \draw[black] (3, 5) -- ++(0, 1);
    
    % Horizontale Linien der Feature-Maps
        \draw[black] (2.5, 1.5) -- ++(1, 0);
        \draw[black] (2.5, 3.5) -- ++(1, 0);
        \draw[black] (2.5, 5.5) -- ++(1, 0);

    % Positionierung der Zahlen der Feature-Maps
        \node[orange, scale = 0.55, font = \bfseries] at (2.75, 5.25) { .45};
        \node[orange, scale = 0.55, font = \bfseries] at (3.25, 5.25) { .87};
        \node[orange, scale = 0.55, font = \bfseries] at (2.75, 5.75) { .90};
        \node[orange, scale = 0.55, font = \bfseries] at (3.25, 5.75) { .65};
        
        \node[orange, scale = 0.55, font = \bfseries] at (2.75, 3.25) { .23};
        \node[orange, scale = 0.55, font = \bfseries] at (3.25, 3.25) { .63};
        \node[orange, scale = 0.55, font = \bfseries] at (2.75, 3.75) { .96};
        \node[orange, scale = 0.55, font = \bfseries] at (3.25, 3.75) { .73};
        
        \node[orange, scale = 0.55, font = \bfseries] at (2.75, 1.25) { .94};
        \node[orange, scale = 0.55, font = \bfseries] at (3.25, 1.25) { .53};
        \node[orange, scale = 0.55, font = \bfseries] at (2.75, 1.75) { .44};
        \node[orange, scale = 0.55, font = \bfseries] at (3.25, 1.75) { .89};
        
    % Größe des Fully-Connected-Layer
        \draw[black] (5, 0.5) rectangle (5.5, 1);
        \draw[black] (5, 1) rectangle (5.5, 1.5);
        \draw[black] (5, 1.5) rectangle (5.5, 2);
        \draw[black] (5, 2) rectangle (5.5, 2.5);
        \draw[black] (5, 2.5) rectangle (5.5, 3);
        \draw[black] (5, 3) rectangle (5.5, 3.5);
        \draw[black] (5, 3.5) rectangle (5.5, 4);
        \draw[black] (5, 4) rectangle (5.5, 4.5);
        \draw[black] (5, 4.5) rectangle (5.5, 5);
        \draw[black] (5, 5) rectangle (5.5, 5.5);
        \draw[black] (5, 5.5) rectangle (5.5, 6);
        \draw[black] (5, 6) rectangle (5.5, 6.5);
    
    % Positionierung der Zahlen im Fully-Connected-Layer
        \node[orange, scale = 0.55, font = \bfseries] at (5.25, 0.75) { .53};
        \node[orange, scale = 0.55, font = \bfseries] at (5.25, 1.25) { .94};
        \node[orange, scale = 0.55, font = \bfseries] at (5.25, 1.75) { .89};
        \node[orange, scale = 0.55, font = \bfseries] at (5.25, 2.25) { .44};
        \node[orange, scale = 0.55, font = \bfseries] at (5.25, 2.75) { .63};
        \node[orange, scale = 0.55, font = \bfseries] at (5.25, 3.25) { .23};
        \node[orange, scale = 0.55, font = \bfseries] at (5.25, 3.75) { .73};
        \node[orange, scale = 0.55, font = \bfseries] at (5.25, 4.25) { .96};
        \node[orange, scale = 0.55, font = \bfseries] at (5.25, 4.75) { .87};
        \node[orange, scale = 0.55, font = \bfseries] at (5.25, 5.25) { .45};
        \node[orange, scale = 0.55, font = \bfseries] at (5.25, 5.75) { .65};
        \node[orange, scale = 0.55, font = \bfseries] at (5.25, 6.25) { .90};
        
    % Verbindungen
        \draw[green, dashed] (3.5, 1) -- (5, 0.5);
        \draw[green, dashed] (3.5, 2) -- (5, 2.5);
        \draw[green, dashed] (3.5, 3) -- (5, 2.5);
        \draw[green, dashed] (3.5, 4) -- (5, 4.5);
        \draw[green, dashed] (3.5, 5) -- (5, 4.5);
        \draw[green, dashed] (3.5, 6) -- (5, 6.5);
    
    % Klassen
        \draw[black] (8, 4.5) rectangle (10, 5.5);
        \draw[black] (8, 1.5) rectangle (10, 2.5);
        \node[black, scale = 1, font = \bfseries] at (9, 5) {X};
        \node[black, scale = 1, font = \bfseries] at (9, 2) {kein X};
    
    % Umrandungen
        \draw[blue, very thick] (5, 6) rectangle (5.5, 6.5);
        \draw[blue, very thick] (5, 4.5) rectangle (5.5, 5);
        \draw[blue, very thick] (5, 4) rectangle (5.5, 4.5);
        \draw[blue, very thick] (5, 1.5) rectangle (5.5, 2);
        \draw[blue, very thick] (5, 1) rectangle (5.5, 1.5);
        \draw[green, very thick] (5, 0.5) rectangle (5.5, 1);
        \draw[green, very thick] (5, 2) rectangle (5.5, 2.5);
        \draw[green, very thick] (5, 2.5) rectangle (5.5, 3);
        \draw[green, very thick] (5, 3) rectangle (5.5, 3.5);
        \draw[green, very thick] (5, 5) rectangle (5.5, 5.5);
    
    % Gewichtungen
        \draw[blue, dashed, ultra thick] (5.5, 6.25) -- (8, 5);
        \draw[green, dashed, thin] (5.5, 6.25) -- (8, 2);
        
        \draw[blue, dashed, thin] (5.5, 5.75) -- (8, 5);
        \draw[green, dashed, thin] (5.5, 5.75) -- (8, 2);
        
        \draw[blue, dashed, thin] (5.5, 5.25) -- (8, 5);
        \draw[green, dashed, ultra thick] (5.5, 5.25) -- (8, 2);
        
        \draw[blue, dashed, ultra thick] (5.5, 4.75) -- (8, 5);
        \draw[green, dashed, thin] (5.5, 4.75) -- (8, 2);
        
        \draw[blue, dashed, ultra thick] (5.5, 4.25) -- (8, 5);
        \draw[green, dashed, thin] (5.5, 4.25) -- (8, 2);
        
        \draw[blue, dashed, thin] (5.5, 3.75) -- (8, 5);
        \draw[green, dashed, thin] (5.5, 3.75) -- (8, 2);
        
        \draw[blue, dashed, thin] (5.5, 3.25) -- (8, 5);
        \draw[green, dashed, ultra thick] (5.5, 3.25) -- (8, 2);
        
        \draw[blue, dashed, thin] (5.5, 2.75) -- (8, 5);
        \draw[green, dashed, ultra thick] (5.5, 2.75) -- (8, 2);
        
        \draw[blue, dashed, thin] (5.5, 2.25) -- (8, 5);
        \draw[green, dashed, ultra thick] (5.5, 2.25) -- (8, 2);
        
        \draw[blue, dashed, ultra thick] (5.5, 1.75) -- (8, 5);
        \draw[green, dashed, thin] (5.5, 1.75) -- (8, 2);
        
        \draw[blue, dashed, ultra thick] (5.5, 1.25) -- (8, 5);
        \draw[green, dashed, thin] (5.5, 1.25) -- (8, 2);
        
        \draw[blue, dashed, thin] (5.5, 0.75) -- (8, 5);
        \draw[green, dashed, ultra thick] (5.5, 0.75) -- (8, 2);
    
    % Wahrscheinlichkeiten
        \node[black, scale = 0.7, font = \bfseries] at (11.5, 5) {0.912 (91.2\%)};
        \node[black, scale = 0.7, font = \bfseries] at (11.5, 2) {0.456 (45.6\%)};
    \end{tikzpicture}
    }
    \caption[Gewichtungen im Fully-Connected-Layer]{Gewichtungen im Fully-Connected-Layer \cite{Wuertz2019}}
    \label{fig:Bild2.8}
\end{figure}

\subsection{AlexNet} \label{sec: alexnet}
Das AlexNet ist ein Pre-Trained CNN und wurde von Alex Krizhevsky in Zusammenarbeit mit Ilya Sutskever and Geoffrey Hinton entwickelt \cite{Wikipedia2022}. Das CNN enthält alle vorher betrachteten Layer-Typen (s. \autoref{sec: convolution} bis \autoref{sec: fully-connected}).\\
Das AlexNet wurde 2006 mit 1.2 Mio. Bildern mit einer Größe von jeweils 227x227x3 trainiert und erreichte eine Fehlerrate von 16.4\% \cite{Kalaiarasi2020}. Das CNN ermöglichte einfachere und schnellere Fortschritte in der Gesichtserkennung. Mittlerweile wurde das CNN durch das leistungsstärkere GoogleLeNet mit 22 Layern abgelöst.\\
Das AlexNet besteht aus fünf Convolution- und drei Pooling-Schichten gefolgt von drei Fully-Connected-Layern. Zwischen den Schichten wird teilweise eine ReLU angewendet \cite{Vasudevan2020}. Die schematische Struktur ist in \autoref{fig:Bild2.9} dargestellt.\\
Das AlexNet wird aufgrund des einfachen Aufbaus im Weiteren  zur Bildverarbeitung zur Vermeidung von Kollisionen verwendet.

\begin{figure}[H]
    \centering
    \scalebox{0.75}
    {
    \begin{tikzpicture}[framed][domain=0:0]
        % Umgebung
            \draw[black] (0,0) rectangle (11, 20);
        
        % Rechtecke für Texte
            \draw[black] (5, 1.25) rectangle (10, 1.75);
            \draw[black] (5, 2.5) rectangle (10, 3);
            \draw[black] (5, 3.75) rectangle (10, 4.25);
            \draw[black] (5, 5) rectangle (10, 6);
            \draw[black] (5, 6.75) rectangle (10, 7.75);
            \draw[black] (5, 8.5) rectangle (10, 9.5);
            \draw[black] (5, 10.25) rectangle (10, 11.25);
            \draw[black] (5, 12) rectangle (10, 13);
            \draw[black] (5, 13.75) rectangle (10, 14.75);
            \draw[black] (5, 15.5) rectangle (10, 16.5);
            \draw[black] (5, 17.25) rectangle (10, 18.25);
            \draw[black] (5, 19) rectangle (10, 19.5);
        
        % Texte
            \node[black, scale = 0.7, font = \bfseries] at (7.5, 1.5) {1000 fully connected neurons};
            \node[black, scale = 0.7, font = \bfseries] at (7.5, 2.75) {4096 fully connected neurons};
            \node[black, scale = 0.7, font = \bfseries] at (7.5, 4) {9216 fully connected neurons};
            \node[black, scale = 0.7, font = \bfseries] at (7.5, 5.7) {max. Pooling with 3x3 kernel};
            \node[black, scale = 0.7, font = \bfseries] at (7.5, 5.3) {2 strides: 6x6x256};
            \node[black, scale = 0.7, font = \bfseries] at (7.5, 7.45) {Convolution with 3x3 kernel};
            \node[black, scale = 0.7, font = \bfseries] at (7.5, 7.05) {1 stride: 13x13x256};
            \node[black, scale = 0.7, font = \bfseries] at (7.5, 9.2) {Convolution with 3x3 kernel};
            \node[black, scale = 0.7, font = \bfseries] at (7.5, 8.8) {1 stride: 13x13x384};
            \node[black, scale = 0.7, font = \bfseries] at (7.5, 10.95) {Convolution with 3x3 kernel};
            \node[black, scale = 0.7, font = \bfseries] at (7.5, 10.55) {1 stride: 13x13x384};
            \node[black, scale = 0.7, font = \bfseries] at (7.5, 12.7) {max. Pooling with 3x3 kernel};
            \node[black, scale = 0.7, font = \bfseries] at (7.5, 12.3) {2 strides: 13x13x256};
            \node[black, scale = 0.7, font = \bfseries] at (7.5, 14.45) {Convolution with 5x5 kernel};
            \node[black, scale = 0.7, font = \bfseries] at (7.5, 14.05) {2 strides: 27x27x256};
            \node[black, scale = 0.7, font = \bfseries] at (7.5, 16.2) {max. Pooling with 3x3 kernel};
            \node[black, scale = 0.7, font = \bfseries] at (7.5, 15.8) {2 strides: 27x27x96};
            \node[black, scale = 0.7, font = \bfseries] at (7.5, 17.95) {Convolution with 11x11 kernel};
            \node[black, scale = 0.7, font = \bfseries] at (7.5, 17.55) {4 strides: 55x55x96};
            \node[black, scale = 0.7, font = \bfseries] at (7.5, 19.25) {Picture: 227x227x3};
        
        % Pfeile
            \draw[->] (7.5, 19) -- (7.5, 18.25);
            \draw[->] (7.5, 17.25) -- (7.5, 16.5);
            \draw[->] (7.5, 15.5) -- (7.5, 14.75);
            \draw[->] (7.5, 13.75) -- (7.5, 13);
            \draw[->] (7.5, 12) -- (7.5, 11.25);
            \draw[->] (7.5, 10.25) -- (7.5, 9.5);
            \draw[->] (7.5, 8.5) -- (7.5, 7.75);
            \draw[->] (7.5, 6.75) -- (7.5, 6);
            \draw[->] (7.5, 5) -- (7.5, 4.25);
            \draw[->] (7.5, 3.75) -- (7.5, 3);
            \draw[->] (7.5, 2.5) -- (7.5, 1.75);
            \draw[->] (7.5, 1.25) -- (7.5, 1.0);
        
        % Zusätzliche Bezeichnungen
            \node[black, scale = 0.7, font = \bfseries] at (8, 16.875) {ReLU};
            \node[black, scale = 0.7, font = \bfseries] at (8, 13.375) {ReLU};
            \node[black, scale = 0.7, font = \bfseries] at (8, 9.875) {ReLU};
            \node[black, scale = 0.7, font = \bfseries] at (8, 8.125) {ReLU};
            \node[black, scale = 0.7, font = \bfseries] at (8, 6.375) {ReLU};
            \node[black, scale = 0.7, font = \bfseries] at (8, 3.375) {ReLU};
            \node[black, scale = 0.7, font = \bfseries] at (8, 2.125) {ReLU};
            \node[black, scale = 0.7, font = \bfseries] at (7.5, 0.5) {Output: 1 of 1000 classes};
        
        % Anzahl der Feature-Maps
            \node[black, scale = 0.7, font = \bfseries] at (3, 17.75) {96 Feature-Maps};
            \node[black, scale = 0.7, font = \bfseries] at (3, 16) {96 Feature-Maps};
            \node[black, scale = 0.7, font = \bfseries] at (3, 14.25) {256 Feature-Maps};
            \node[black, scale = 0.7, font = \bfseries] at (3, 12.5) {256 Feature-Maps};
            \node[black, scale = 0.7, font = \bfseries] at (3, 10.75) {384 Feature-Maps};
            \node[black, scale = 0.7, font = \bfseries] at (3, 9) {384 Feature-Maps};
            \node[black, scale = 0.7, font = \bfseries] at (3, 7.25) {256 Feature-Maps};
            \node[black, scale = 0.7, font = \bfseries] at (3, 5.5) {256 Feature-Maps};
            \draw[decorate , decoration = {brace, amplitude = 5pt, mirror}] (4.5, 4.25) -- ++(0, -3);
            \node[black, scale = 0.7, font = \bfseries] at (2.8, 2.9) {3 Fully-Connected-};
            \node[black, scale = 0.7, font = \bfseries] at (2.8, 2.5) {Layer};
    \end{tikzpicture}
    }
    \caption[Schematischer Aufbau des AlexNet]{Schematischer Aufbau des AlexNet \cite{Wikipedia2022}}
    \label{fig:Bild2.9}
\end{figure}

\subsection{Transfer-Learning} \label{sec: transfer-learning}
Um das AlexNet aus \autoref{sec: alexnet} sinnvoll nutzen zu können, wird die Maschine-Learning-Technik des \textbf{Transfer-Learning} angewandt. Beim Transfer-Learning wird ein bereits \textbf{vortrainiertes neuronales Netz} (z.B. CNN oder ResNet) auf ein \textbf{ähnliches Problem} angewendet (z.B: Bild- und Textverarbeitung) \cite{Woelki2020}. Die Anwendung dessen ist dann sinnvoll, wenn nur ein kleiner eigener Datensatz mit wenigen Klassen zur Verfügung steht. Durch den Transfer werden sämtliche Skills und Eigenschaften übernommen und auf das eigene Problem adaptiert (\autoref{fig:Bild2.10}). Dies reduziert den eigenen Ressourceneinsatz, ermöglicht eine schnellere Erstellung und Anwendung auf ein Problem und erhöht die Modellqualität \cite{Woelki2020}.\\

\begin{figure}[H]
    \centering
    \scalebox{0.95}{%
    \begin{tikzpicture}[framed][domain=0:0]
        % Umgebung
        \draw[black] (0, 0) rectangle (15, 7.5);
        
        % Kästchen
        \draw[black] (0.5, 1) rectangle (2.5, 2);
        \draw[black] (4.5, 1) rectangle (6.5, 2);
        \draw[black] (8.5, 1) rectangle (10.5, 2);
        \draw[black] (12.5, 1) rectangle (14.5, 2);
        \draw[black] (0.5, 5) rectangle (2.5, 6);
        \draw[black] (4.5, 5) rectangle (6.5, 6);
        \draw[black] (8.5, 5) rectangle (10.5, 6);
        \draw[black] (12.5, 5) rectangle (14.5, 6);
        
        % Pfeile
        \draw[->] (2.5, 1.5) -- (4.5, 1.5);
        \draw[->] (6.5, 1.5) -- (8.5, 1.5);
        \draw[->] (10.5, 1.5) -- (12.5, 1.5);
        \draw[->] (2.5, 5.5) -- (4.5, 5.5);
        \draw[->] (6.5, 5.5) -- (8.5, 5.5);
        \draw[->] (10.5, 5.5) -- (12.5, 5.5);
        \draw[->] (5.5, 5) -- (5.5, 2);
        
        % Umrandungen
        \draw[blue, thick] (4, 0.5) rectangle (11, 2.5);
        \draw[blue, thick] (4, 4.5) rectangle (11, 6.5);
        
        % Texte
        \node[black, scale = 0.7, font = \bfseries] at (1.5, 1.5) {Data2};
        \node[black, scale = 0.7, font = \bfseries] at (5.5, 1.5) {Model1};
        \node[black, scale = 0.7, font = \bfseries] at (9.5, 1.5) {New Head};
        \node[black, scale = 0.7, font = \bfseries] at (13.5, 1.5) {Predictions2};
        \node[black, scale = 0.7, font = \bfseries] at (1.5, 5.5) {Data1};
        \node[black, scale = 0.7, font = \bfseries] at (5.5, 5.5) {Model1};
        \node[black, scale = 0.7, font = \bfseries] at (9.5, 5.5) {Head};
        \node[black, scale = 0.7, font = \bfseries] at (13.5, 5.5) {Predictions1};
        \node[black, scale = 0.7, font = \bfseries] at (3.75, 3.5) {Knowledge Transfer};
        
        % Überschriften
        \node[blue, scale = 0.9, font = \bfseries] at (7.5, 3) {Task 2};
        \node[blue, scale = 0.9, font = \bfseries] at (7.5, 7) {Task 1};

    \end{tikzpicture}
    }
    \caption[Visualisierung des Transfer-Learning]{Visualsierung des Transfer-Learning \cite{Sonwane2020}}
    \label{fig:Bild2.10}
\end{figure}

Das Transfer-Learning wird in die Bereiche \textbf{Pre-Training} und \textbf{Fine-Tuning} unterschieden. Beim Pre-Training werden die Feauture-Maps des neuronalen Netzes unverändert übernommen und lediglich die Zuordnung der Klassen (Predictions) an die Klassen des neuen Problems angepasst und trainiert \cite{Woelki2020}.\\
Beim Fine-Tuning werden zusätzlich die Feature-Maps entsprechend auf das spezifische Problem angepasst. Hierzu werden größere eigene Datenmengen benötigt \cite{Woelki2020}.\\

\textbf{Aufgrund des geringen eigenen Datensatzes wird das Pre-Training des AlexNets bevorzugt angewendet.}
    

